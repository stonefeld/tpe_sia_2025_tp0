\documentclass[11pt]{article}

    \usepackage[breakable]{tcolorbox}
    \usepackage{parskip} % Stop auto-indenting (to mimic markdown behaviour)


    % Basic figure setup, for now with no caption control since it's done
    % automatically by Pandoc (which extracts ![](path) syntax from Markdown).
    \usepackage{graphicx}
    % Keep aspect ratio if custom image width or height is specified
    \setkeys{Gin}{keepaspectratio}
    % Maintain compatibility with old templates. Remove in nbconvert 6.0
    \let\Oldincludegraphics\includegraphics
    % Ensure that by default, figures have no caption (until we provide a
    % proper Figure object with a Caption API and a way to capture that
    % in the conversion process - todo).
    \usepackage{caption}
    \DeclareCaptionFormat{nocaption}{}
    \captionsetup{format=nocaption,aboveskip=0pt,belowskip=0pt}

    \usepackage{float}
    \floatplacement{figure}{H} % forces figures to be placed at the correct location
    \usepackage{xcolor} % Allow colors to be defined
    \usepackage{enumerate} % Needed for markdown enumerations to work
    \usepackage{geometry} % Used to adjust the document margins
    \usepackage{amsmath} % Equations
    \usepackage{amssymb} % Equations
    \usepackage{textcomp} % defines textquotesingle
    % Hack from http://tex.stackexchange.com/a/47451/13684:
    \AtBeginDocument{%
        \def\PYZsq{\textquotesingle}% Upright quotes in Pygmentized code
    }
    \usepackage{upquote} % Upright quotes for verbatim code
    \usepackage{eurosym} % defines \euro

    \usepackage{iftex}
    \ifPDFTeX
        \usepackage[T1]{fontenc}
        \IfFileExists{alphabeta.sty}{
              \usepackage{alphabeta}
          }{
              \usepackage[mathletters]{ucs}
              \usepackage[utf8x]{inputenc}
          }
    \else
        \usepackage{fontspec}
        \usepackage{unicode-math}
    \fi

    \usepackage{fancyvrb} % verbatim replacement that allows latex
    \usepackage{grffile} % extends the file name processing of package graphics
                         % to support a larger range
    \makeatletter % fix for old versions of grffile with XeLaTeX
    \@ifpackagelater{grffile}{2019/11/01}
    {
      % Do nothing on new versions
    }
    {
      \def\Gread@@xetex#1{%
        \IfFileExists{"\Gin@base".bb}%
        {\Gread@eps{\Gin@base.bb}}%
        {\Gread@@xetex@aux#1}%
      }
    }
    \makeatother
    \usepackage[Export]{adjustbox} % Used to constrain images to a maximum size
    \adjustboxset{max size={0.9\linewidth}{0.9\paperheight}}

    % The hyperref package gives us a pdf with properly built
    % internal navigation ('pdf bookmarks' for the table of contents,
    % internal cross-reference links, web links for URLs, etc.)
    \usepackage{hyperref}
    % The default LaTeX title has an obnoxious amount of whitespace. By default,
    % titling removes some of it. It also provides customization options.
    \usepackage{titling}
    \usepackage{longtable} % longtable support required by pandoc >1.10
    \usepackage{booktabs}  % table support for pandoc > 1.12.2
    \usepackage{array}     % table support for pandoc >= 2.11.3
    \usepackage{calc}      % table minipage width calculation for pandoc >= 2.11.1
    \usepackage[inline]{enumitem} % IRkernel/repr support (it uses the enumerate* environment)
    \usepackage[normalem]{ulem} % ulem is needed to support strikethroughs (\sout)
                                % normalem makes italics be italics, not underlines
    \usepackage{soul}      % strikethrough (\st) support for pandoc >= 3.0.0
    \usepackage{mathrsfs}

    \usepackage{titlesec}
    \renewcommand*\contentsname{Tabla de contenidos}



    % Colors for the hyperref package
    \definecolor{urlcolor}{rgb}{0,.145,.698}
    \definecolor{linkcolor}{rgb}{.71,0.21,0.01}
    \definecolor{citecolor}{rgb}{.12,.54,.11}

    % ANSI colors
    \definecolor{ansi-black}{HTML}{3E424D}
    \definecolor{ansi-black-intense}{HTML}{282C36}
    \definecolor{ansi-red}{HTML}{E75C58}
    \definecolor{ansi-red-intense}{HTML}{B22B31}
    \definecolor{ansi-green}{HTML}{00A250}
    \definecolor{ansi-green-intense}{HTML}{007427}
    \definecolor{ansi-yellow}{HTML}{DDB62B}
    \definecolor{ansi-yellow-intense}{HTML}{B27D12}
    \definecolor{ansi-blue}{HTML}{208FFB}
    \definecolor{ansi-blue-intense}{HTML}{0065CA}
    \definecolor{ansi-magenta}{HTML}{D160C4}
    \definecolor{ansi-magenta-intense}{HTML}{A03196}
    \definecolor{ansi-cyan}{HTML}{60C6C8}
    \definecolor{ansi-cyan-intense}{HTML}{258F8F}
    \definecolor{ansi-white}{HTML}{C5C1B4}
    \definecolor{ansi-white-intense}{HTML}{A1A6B2}
    \definecolor{ansi-default-inverse-fg}{HTML}{FFFFFF}
    \definecolor{ansi-default-inverse-bg}{HTML}{000000}

    % common color for the border for error outputs.
    \definecolor{outerrorbackground}{HTML}{FFDFDF}

    % commands and environments needed by pandoc snippets
    % extracted from the output of `pandoc -s`
    \providecommand{\tightlist}{%
      \setlength{\itemsep}{0pt}\setlength{\parskip}{0pt}}
    \DefineVerbatimEnvironment{Highlighting}{Verbatim}{commandchars=\\\{\}}
    % Add ',fontsize=\small' for more characters per line
    \newenvironment{Shaded}{}{}
    \newcommand{\KeywordTok}[1]{\textcolor[rgb]{0.00,0.44,0.13}{\textbf{{#1}}}}
    \newcommand{\DataTypeTok}[1]{\textcolor[rgb]{0.56,0.13,0.00}{{#1}}}
    \newcommand{\DecValTok}[1]{\textcolor[rgb]{0.25,0.63,0.44}{{#1}}}
    \newcommand{\BaseNTok}[1]{\textcolor[rgb]{0.25,0.63,0.44}{{#1}}}
    \newcommand{\FloatTok}[1]{\textcolor[rgb]{0.25,0.63,0.44}{{#1}}}
    \newcommand{\CharTok}[1]{\textcolor[rgb]{0.25,0.44,0.63}{{#1}}}
    \newcommand{\StringTok}[1]{\textcolor[rgb]{0.25,0.44,0.63}{{#1}}}
    \newcommand{\CommentTok}[1]{\textcolor[rgb]{0.38,0.63,0.69}{\textit{{#1}}}}
    \newcommand{\OtherTok}[1]{\textcolor[rgb]{0.00,0.44,0.13}{{#1}}}
    \newcommand{\AlertTok}[1]{\textcolor[rgb]{1.00,0.00,0.00}{\textbf{{#1}}}}
    \newcommand{\FunctionTok}[1]{\textcolor[rgb]{0.02,0.16,0.49}{{#1}}}
    \newcommand{\RegionMarkerTok}[1]{{#1}}
    \newcommand{\ErrorTok}[1]{\textcolor[rgb]{1.00,0.00,0.00}{\textbf{{#1}}}}
    \newcommand{\NormalTok}[1]{{#1}}

    % Additional commands for more recent versions of Pandoc
    \newcommand{\ConstantTok}[1]{\textcolor[rgb]{0.53,0.00,0.00}{{#1}}}
    \newcommand{\SpecialCharTok}[1]{\textcolor[rgb]{0.25,0.44,0.63}{{#1}}}
    \newcommand{\VerbatimStringTok}[1]{\textcolor[rgb]{0.25,0.44,0.63}{{#1}}}
    \newcommand{\SpecialStringTok}[1]{\textcolor[rgb]{0.73,0.40,0.53}{{#1}}}
    \newcommand{\ImportTok}[1]{{#1}}
    \newcommand{\DocumentationTok}[1]{\textcolor[rgb]{0.73,0.13,0.13}{\textit{{#1}}}}
    \newcommand{\AnnotationTok}[1]{\textcolor[rgb]{0.38,0.63,0.69}{\textbf{\textit{{#1}}}}}
    \newcommand{\CommentVarTok}[1]{\textcolor[rgb]{0.38,0.63,0.69}{\textbf{\textit{{#1}}}}}
    \newcommand{\VariableTok}[1]{\textcolor[rgb]{0.10,0.09,0.49}{{#1}}}
    \newcommand{\ControlFlowTok}[1]{\textcolor[rgb]{0.00,0.44,0.13}{\textbf{{#1}}}}
    \newcommand{\OperatorTok}[1]{\textcolor[rgb]{0.40,0.40,0.40}{{#1}}}
    \newcommand{\BuiltInTok}[1]{{#1}}
    \newcommand{\ExtensionTok}[1]{{#1}}
    \newcommand{\PreprocessorTok}[1]{\textcolor[rgb]{0.74,0.48,0.00}{{#1}}}
    \newcommand{\AttributeTok}[1]{\textcolor[rgb]{0.49,0.56,0.16}{{#1}}}
    \newcommand{\InformationTok}[1]{\textcolor[rgb]{0.38,0.63,0.69}{\textbf{\textit{{#1}}}}}
    \newcommand{\WarningTok}[1]{\textcolor[rgb]{0.38,0.63,0.69}{\textbf{\textit{{#1}}}}}
    \makeatletter
    \newsavebox\pandoc@box
    \newcommand*\pandocbounded[1]{%
      \sbox\pandoc@box{#1}%
      % scaling factors for width and height
      \Gscale@div\@tempa\textheight{\dimexpr\ht\pandoc@box+\dp\pandoc@box\relax}%
      \Gscale@div\@tempb\linewidth{\wd\pandoc@box}%
      % select the smaller of both
      \ifdim\@tempb\p@<\@tempa\p@
        \let\@tempa\@tempb
      \fi
      % scaling accordingly (\@tempa < 1)
      \ifdim\@tempa\p@<\p@
        \scalebox{\@tempa}{\usebox\pandoc@box}%
      % scaling not needed, use as it is
      \else
        \usebox{\pandoc@box}%
      \fi
    }
    \makeatother

    % Define a nice break command that doesn't care if a line doesn't already
    % exist.
    \def\br{\hspace*{\fill} \\* }
    % Math Jax compatibility definitions
    \def\gt{>}
    \def\lt{<}
    \let\Oldtex\TeX
    \let\Oldlatex\LaTeX
    \renewcommand{\TeX}{\textrm{\Oldtex}}
    \renewcommand{\LaTeX}{\textrm{\Oldlatex}}







% Pygments definitions
\makeatletter
\def\PY@reset{\let\PY@it=\relax \let\PY@bf=\relax%
    \let\PY@ul=\relax \let\PY@tc=\relax%
    \let\PY@bc=\relax \let\PY@ff=\relax}
\def\PY@tok#1{\csname PY@tok@#1\endcsname}
\def\PY@toks#1+{\ifx\relax#1\empty\else%
    \PY@tok{#1}\expandafter\PY@toks\fi}
\def\PY@do#1{\PY@bc{\PY@tc{\PY@ul{%
    \PY@it{\PY@bf{\PY@ff{#1}}}}}}}
\def\PY#1#2{\PY@reset\PY@toks#1+\relax+\PY@do{#2}}

\@namedef{PY@tok@w}{\def\PY@tc##1{\textcolor[rgb]{0.73,0.73,0.73}{##1}}}
\@namedef{PY@tok@c}{\let\PY@it=\textit\def\PY@tc##1{\textcolor[rgb]{0.24,0.48,0.48}{##1}}}
\@namedef{PY@tok@cp}{\def\PY@tc##1{\textcolor[rgb]{0.61,0.40,0.00}{##1}}}
\@namedef{PY@tok@k}{\let\PY@bf=\textbf\def\PY@tc##1{\textcolor[rgb]{0.00,0.50,0.00}{##1}}}
\@namedef{PY@tok@kp}{\def\PY@tc##1{\textcolor[rgb]{0.00,0.50,0.00}{##1}}}
\@namedef{PY@tok@kt}{\def\PY@tc##1{\textcolor[rgb]{0.69,0.00,0.25}{##1}}}
\@namedef{PY@tok@o}{\def\PY@tc##1{\textcolor[rgb]{0.40,0.40,0.40}{##1}}}
\@namedef{PY@tok@ow}{\let\PY@bf=\textbf\def\PY@tc##1{\textcolor[rgb]{0.67,0.13,1.00}{##1}}}
\@namedef{PY@tok@nb}{\def\PY@tc##1{\textcolor[rgb]{0.00,0.50,0.00}{##1}}}
\@namedef{PY@tok@nf}{\def\PY@tc##1{\textcolor[rgb]{0.00,0.00,1.00}{##1}}}
\@namedef{PY@tok@nc}{\let\PY@bf=\textbf\def\PY@tc##1{\textcolor[rgb]{0.00,0.00,1.00}{##1}}}
\@namedef{PY@tok@nn}{\let\PY@bf=\textbf\def\PY@tc##1{\textcolor[rgb]{0.00,0.00,1.00}{##1}}}
\@namedef{PY@tok@ne}{\let\PY@bf=\textbf\def\PY@tc##1{\textcolor[rgb]{0.80,0.25,0.22}{##1}}}
\@namedef{PY@tok@nv}{\def\PY@tc##1{\textcolor[rgb]{0.10,0.09,0.49}{##1}}}
\@namedef{PY@tok@no}{\def\PY@tc##1{\textcolor[rgb]{0.53,0.00,0.00}{##1}}}
\@namedef{PY@tok@nl}{\def\PY@tc##1{\textcolor[rgb]{0.46,0.46,0.00}{##1}}}
\@namedef{PY@tok@ni}{\let\PY@bf=\textbf\def\PY@tc##1{\textcolor[rgb]{0.44,0.44,0.44}{##1}}}
\@namedef{PY@tok@na}{\def\PY@tc##1{\textcolor[rgb]{0.41,0.47,0.13}{##1}}}
\@namedef{PY@tok@nt}{\let\PY@bf=\textbf\def\PY@tc##1{\textcolor[rgb]{0.00,0.50,0.00}{##1}}}
\@namedef{PY@tok@nd}{\def\PY@tc##1{\textcolor[rgb]{0.67,0.13,1.00}{##1}}}
\@namedef{PY@tok@s}{\def\PY@tc##1{\textcolor[rgb]{0.73,0.13,0.13}{##1}}}
\@namedef{PY@tok@sd}{\let\PY@it=\textit\def\PY@tc##1{\textcolor[rgb]{0.73,0.13,0.13}{##1}}}
\@namedef{PY@tok@si}{\let\PY@bf=\textbf\def\PY@tc##1{\textcolor[rgb]{0.64,0.35,0.47}{##1}}}
\@namedef{PY@tok@se}{\let\PY@bf=\textbf\def\PY@tc##1{\textcolor[rgb]{0.67,0.36,0.12}{##1}}}
\@namedef{PY@tok@sr}{\def\PY@tc##1{\textcolor[rgb]{0.64,0.35,0.47}{##1}}}
\@namedef{PY@tok@ss}{\def\PY@tc##1{\textcolor[rgb]{0.10,0.09,0.49}{##1}}}
\@namedef{PY@tok@sx}{\def\PY@tc##1{\textcolor[rgb]{0.00,0.50,0.00}{##1}}}
\@namedef{PY@tok@m}{\def\PY@tc##1{\textcolor[rgb]{0.40,0.40,0.40}{##1}}}
\@namedef{PY@tok@gh}{\let\PY@bf=\textbf\def\PY@tc##1{\textcolor[rgb]{0.00,0.00,0.50}{##1}}}
\@namedef{PY@tok@gu}{\let\PY@bf=\textbf\def\PY@tc##1{\textcolor[rgb]{0.50,0.00,0.50}{##1}}}
\@namedef{PY@tok@gd}{\def\PY@tc##1{\textcolor[rgb]{0.63,0.00,0.00}{##1}}}
\@namedef{PY@tok@gi}{\def\PY@tc##1{\textcolor[rgb]{0.00,0.52,0.00}{##1}}}
\@namedef{PY@tok@gr}{\def\PY@tc##1{\textcolor[rgb]{0.89,0.00,0.00}{##1}}}
\@namedef{PY@tok@ge}{\let\PY@it=\textit}
\@namedef{PY@tok@gs}{\let\PY@bf=\textbf}
\@namedef{PY@tok@ges}{\let\PY@bf=\textbf\let\PY@it=\textit}
\@namedef{PY@tok@gp}{\let\PY@bf=\textbf\def\PY@tc##1{\textcolor[rgb]{0.00,0.00,0.50}{##1}}}
\@namedef{PY@tok@go}{\def\PY@tc##1{\textcolor[rgb]{0.44,0.44,0.44}{##1}}}
\@namedef{PY@tok@gt}{\def\PY@tc##1{\textcolor[rgb]{0.00,0.27,0.87}{##1}}}
\@namedef{PY@tok@err}{\def\PY@bc##1{{\setlength{\fboxsep}{\string -\fboxrule}\fcolorbox[rgb]{1.00,0.00,0.00}{1,1,1}{\strut ##1}}}}
\@namedef{PY@tok@kc}{\let\PY@bf=\textbf\def\PY@tc##1{\textcolor[rgb]{0.00,0.50,0.00}{##1}}}
\@namedef{PY@tok@kd}{\let\PY@bf=\textbf\def\PY@tc##1{\textcolor[rgb]{0.00,0.50,0.00}{##1}}}
\@namedef{PY@tok@kn}{\let\PY@bf=\textbf\def\PY@tc##1{\textcolor[rgb]{0.00,0.50,0.00}{##1}}}
\@namedef{PY@tok@kr}{\let\PY@bf=\textbf\def\PY@tc##1{\textcolor[rgb]{0.00,0.50,0.00}{##1}}}
\@namedef{PY@tok@bp}{\def\PY@tc##1{\textcolor[rgb]{0.00,0.50,0.00}{##1}}}
\@namedef{PY@tok@fm}{\def\PY@tc##1{\textcolor[rgb]{0.00,0.00,1.00}{##1}}}
\@namedef{PY@tok@vc}{\def\PY@tc##1{\textcolor[rgb]{0.10,0.09,0.49}{##1}}}
\@namedef{PY@tok@vg}{\def\PY@tc##1{\textcolor[rgb]{0.10,0.09,0.49}{##1}}}
\@namedef{PY@tok@vi}{\def\PY@tc##1{\textcolor[rgb]{0.10,0.09,0.49}{##1}}}
\@namedef{PY@tok@vm}{\def\PY@tc##1{\textcolor[rgb]{0.10,0.09,0.49}{##1}}}
\@namedef{PY@tok@sa}{\def\PY@tc##1{\textcolor[rgb]{0.73,0.13,0.13}{##1}}}
\@namedef{PY@tok@sb}{\def\PY@tc##1{\textcolor[rgb]{0.73,0.13,0.13}{##1}}}
\@namedef{PY@tok@sc}{\def\PY@tc##1{\textcolor[rgb]{0.73,0.13,0.13}{##1}}}
\@namedef{PY@tok@dl}{\def\PY@tc##1{\textcolor[rgb]{0.73,0.13,0.13}{##1}}}
\@namedef{PY@tok@s2}{\def\PY@tc##1{\textcolor[rgb]{0.73,0.13,0.13}{##1}}}
\@namedef{PY@tok@sh}{\def\PY@tc##1{\textcolor[rgb]{0.73,0.13,0.13}{##1}}}
\@namedef{PY@tok@s1}{\def\PY@tc##1{\textcolor[rgb]{0.73,0.13,0.13}{##1}}}
\@namedef{PY@tok@mb}{\def\PY@tc##1{\textcolor[rgb]{0.40,0.40,0.40}{##1}}}
\@namedef{PY@tok@mf}{\def\PY@tc##1{\textcolor[rgb]{0.40,0.40,0.40}{##1}}}
\@namedef{PY@tok@mh}{\def\PY@tc##1{\textcolor[rgb]{0.40,0.40,0.40}{##1}}}
\@namedef{PY@tok@mi}{\def\PY@tc##1{\textcolor[rgb]{0.40,0.40,0.40}{##1}}}
\@namedef{PY@tok@il}{\def\PY@tc##1{\textcolor[rgb]{0.40,0.40,0.40}{##1}}}
\@namedef{PY@tok@mo}{\def\PY@tc##1{\textcolor[rgb]{0.40,0.40,0.40}{##1}}}
\@namedef{PY@tok@ch}{\let\PY@it=\textit\def\PY@tc##1{\textcolor[rgb]{0.24,0.48,0.48}{##1}}}
\@namedef{PY@tok@cm}{\let\PY@it=\textit\def\PY@tc##1{\textcolor[rgb]{0.24,0.48,0.48}{##1}}}
\@namedef{PY@tok@cpf}{\let\PY@it=\textit\def\PY@tc##1{\textcolor[rgb]{0.24,0.48,0.48}{##1}}}
\@namedef{PY@tok@c1}{\let\PY@it=\textit\def\PY@tc##1{\textcolor[rgb]{0.24,0.48,0.48}{##1}}}
\@namedef{PY@tok@cs}{\let\PY@it=\textit\def\PY@tc##1{\textcolor[rgb]{0.24,0.48,0.48}{##1}}}

\def\PYZbs{\char`\\}
\def\PYZus{\char`\_}
\def\PYZob{\char`\{}
\def\PYZcb{\char`\}}
\def\PYZca{\char`\^}
\def\PYZam{\char`\&}
\def\PYZlt{\char`\<}
\def\PYZgt{\char`\>}
\def\PYZsh{\char`\#}
\def\PYZpc{\char`\%}
\def\PYZdl{\char`\$}
\def\PYZhy{\char`\-}
\def\PYZsq{\char`\'}
\def\PYZdq{\char`\"}
\def\PYZti{\char`\~}
% for compatibility with earlier versions
\def\PYZat{@}
\def\PYZlb{[}
\def\PYZrb{]}
\makeatother


    % For linebreaks inside Verbatim environment from package fancyvrb.
    \makeatletter
        \newbox\Wrappedcontinuationbox
        \newbox\Wrappedvisiblespacebox
        \newcommand*\Wrappedvisiblespace {\textcolor{red}{\textvisiblespace}}
        \newcommand*\Wrappedcontinuationsymbol {\textcolor{red}{\llap{\tiny$\m@th\hookrightarrow$}}}
        \newcommand*\Wrappedcontinuationindent {3ex }
        \newcommand*\Wrappedafterbreak {\kern\Wrappedcontinuationindent\copy\Wrappedcontinuationbox}
        % Take advantage of the already applied Pygments mark-up to insert
        % potential linebreaks for TeX processing.
        %        {, <, #, %, $, ' and ": go to next line.
        %        _, }, ^, &, >, - and ~: stay at end of broken line.
        % Use of \textquotesingle for straight quote.
        \newcommand*\Wrappedbreaksatspecials {%
            \def\PYGZus{\discretionary{\char`\_}{\Wrappedafterbreak}{\char`\_}}%
            \def\PYGZob{\discretionary{}{\Wrappedafterbreak\char`\{}{\char`\{}}%
            \def\PYGZcb{\discretionary{\char`\}}{\Wrappedafterbreak}{\char`\}}}%
            \def\PYGZca{\discretionary{\char`\^}{\Wrappedafterbreak}{\char`\^}}%
            \def\PYGZam{\discretionary{\char`\&}{\Wrappedafterbreak}{\char`\&}}%
            \def\PYGZlt{\discretionary{}{\Wrappedafterbreak\char`\<}{\char`\<}}%
            \def\PYGZgt{\discretionary{\char`\>}{\Wrappedafterbreak}{\char`\>}}%
            \def\PYGZsh{\discretionary{}{\Wrappedafterbreak\char`\#}{\char`\#}}%
            \def\PYGZpc{\discretionary{}{\Wrappedafterbreak\char`\%}{\char`\%}}%
            \def\PYGZdl{\discretionary{}{\Wrappedafterbreak\char`\$}{\char`\$}}%
            \def\PYGZhy{\discretionary{\char`\-}{\Wrappedafterbreak}{\char`\-}}%
            \def\PYGZsq{\discretionary{}{\Wrappedafterbreak\textquotesingle}{\textquotesingle}}%
            \def\PYGZdq{\discretionary{}{\Wrappedafterbreak\char`\"}{\char`\"}}%
            \def\PYGZti{\discretionary{\char`\~}{\Wrappedafterbreak}{\char`\~}}%
        }
        % Some characters . , ; ? ! / are not pygmentized.
        % This macro makes them "active" and they will insert potential linebreaks
        \newcommand*\Wrappedbreaksatpunct {%
            \lccode`\~`\.\lowercase{\def~}{\discretionary{\hbox{\char`\.}}{\Wrappedafterbreak}{\hbox{\char`\.}}}%
            \lccode`\~`\,\lowercase{\def~}{\discretionary{\hbox{\char`\,}}{\Wrappedafterbreak}{\hbox{\char`\,}}}%
            \lccode`\~`\;\lowercase{\def~}{\discretionary{\hbox{\char`\;}}{\Wrappedafterbreak}{\hbox{\char`\;}}}%
            \lccode`\~`\:\lowercase{\def~}{\discretionary{\hbox{\char`\:}}{\Wrappedafterbreak}{\hbox{\char`\:}}}%
            \lccode`\~`\?\lowercase{\def~}{\discretionary{\hbox{\char`\?}}{\Wrappedafterbreak}{\hbox{\char`\?}}}%
            \lccode`\~`\!\lowercase{\def~}{\discretionary{\hbox{\char`\!}}{\Wrappedafterbreak}{\hbox{\char`\!}}}%
            \lccode`\~`\/\lowercase{\def~}{\discretionary{\hbox{\char`\/}}{\Wrappedafterbreak}{\hbox{\char`\/}}}%
            \catcode`\.\active
            \catcode`\,\active
            \catcode`\;\active
            \catcode`\:\active
            \catcode`\?\active
            \catcode`\!\active
            \catcode`\/\active
            \lccode`\~`\~
        }
    \makeatother

    \let\OriginalVerbatim=\Verbatim
    \makeatletter
    \renewcommand{\Verbatim}[1][1]{%
        %\parskip\z@skip
        \sbox\Wrappedcontinuationbox {\Wrappedcontinuationsymbol}%
        \sbox\Wrappedvisiblespacebox {\FV@SetupFont\Wrappedvisiblespace}%
        \def\FancyVerbFormatLine ##1{\hsize\linewidth
            \vtop{\raggedright\hyphenpenalty\z@\exhyphenpenalty\z@
                \doublehyphendemerits\z@\finalhyphendemerits\z@
                \strut ##1\strut}%
        }%
        % If the linebreak is at a space, the latter will be displayed as visible
        % space at end of first line, and a continuation symbol starts next line.
        % Stretch/shrink are however usually zero for typewriter font.
        \def\FV@Space {%
            \nobreak\hskip\z@ plus\fontdimen3\font minus\fontdimen4\font
            \discretionary{\copy\Wrappedvisiblespacebox}{\Wrappedafterbreak}
            {\kern\fontdimen2\font}%
        }%

        % Allow breaks at special characters using \PYG... macros.
        \Wrappedbreaksatspecials
        % Breaks at punctuation characters . , ; ? ! and / need catcode=\active
        \OriginalVerbatim[#1,codes*=\Wrappedbreaksatpunct]%
    }
    \makeatother

    % Exact colors from NB
    \definecolor{incolor}{HTML}{303F9F}
    \definecolor{outcolor}{HTML}{D84315}
    \definecolor{cellborder}{HTML}{CFCFCF}
    \definecolor{cellbackground}{HTML}{F7F7F7}

    % prompt
    \makeatletter
    \newcommand{\boxspacing}{\kern\kvtcb@left@rule\kern\kvtcb@boxsep}
    \makeatother
    \newcommand{\prompt}[4]{
        {\ttfamily\llap{{\color{#2}[#3]:\hspace{3pt}#4}}\vspace{-\baselineskip}}
    }



    % Prevent overflowing lines due to hard-to-break entities
    \sloppy
    % Setup hyperref package
    \hypersetup{
      breaklinks=true,  % so long urls are correctly broken across lines
      colorlinks=true,
      urlcolor=urlcolor,
      linkcolor=linkcolor,
      citecolor=citecolor,
      }
    % Slightly bigger margins than the latex defaults

    \geometry{verbose,tmargin=1in,bmargin=1in,lmargin=1in,rmargin=1in}

\usepackage{fancyhdr}
\pagestyle{fancy}
\lhead{72.27 - Sistemas de Inteligencia Artificial}
\rhead{Grupo 6}
\cfoot{\thepage}
\renewcommand{\headrulewidth}{0.4pt}
\renewcommand{\footrulewidth}{0.4pt}

\begin{document}

\begin{titlepage}
  \centering
  \includegraphics[width=0.25\textwidth]{itba_logo}\par\vspace{1cm}
  {\textsc{Instituto Tecnológico de Buenos Aires} \par}
    \vspace{1cm}
    {\Large \textsc{TP 0}\par}
    \vspace{1.5cm}
    {\huge\bfseries Introducción al Análisis de Datos\par}
    \vspace{2cm}
    {\Large\itshape Boullosa Gutierrez, Juan Cruz - 63414\\
	Deyheralde, Ben - 63559\\
	Freire, Tomás - 62027\\
	Stanfield, Theo - 63403\par}
    \vfill
    Sistema de Inteligencia Artificial - 72.27
    \vfill
    {\large Primer cuatrimestre 2025 - Grupo 6\par}
\end{titlepage}

{
  \hypersetup{linkcolor=black}
  \tableofcontents
}

\newpage




\section{Preguntas Guía}\label{preguntas-guuxeda}

Se desea evaluar qué factores influyen en la captura de un Pokemon
teniendo en cuenta los parámetros del mismo junto con la pokebola
utilizada. Para ello se deberán generar distintos gráficos que
representen las respuestas a las siguientes preguntas guía:

\begin{enumerate}
\def\labelenumi{\arabic{enumi}.}
\tightlist
\item
  Acerca de las pokebolas:

  \begin{enumerate}
  \def\labelenumii{\arabic{enumii}.}
  \tightlist
  \item
    Ejecutando la función 100 veces, para cada Pokemon en condiciones
    ideales (HP: 100\%, LVL 100) ¿Cuál es la probabilidad de capturar
    promedio para cada pokebola?
  \item
    ¿Es cierto que algunas pokebolas son más o menos efectivas
    dependiendo de propiedades intrínsecas de cada Pokemon? Justificar.
    \begin{quote}
      \textgreater{} \textbf{Sugerencia}: Comparar efectividad
      (\emph{success/total\_attempts}) como proporción de la efectividad
      de la Pokebola básica para cada Pokemon
    \end{quote}
  \end{enumerate}
\item
  Acerca del estado del Pokemon:

  \begin{enumerate}
  \def\labelenumii{\arabic{enumii}.}
  \item
    ¿Las condiciones de salud tienen algún efecto sobre la efectividad
    de la captura? Si es ası́, ¿Cuál es más o menos efectiva?
  \item
    ¿Cómo afectan los puntos de vida a la efectividad de la captura?
    \begin{quote}
      \textgreater{} \textbf{Sugerencia}: Elegir uno o dos Pokemones y
      manteniendo el resto de los parámetros constantes, calcular la
      probabilidad de captura para distintos HP \%
    \end{quote}
  \item
    ¿Qué parámetros son los que más afectan la probabilidad de captura?
  \item
    Teniendo en cuenta uno o dos pokemones distintos: ¿Qué combinación
    de condiciones (propiedades mutables) y pokebola conviene utilizar
    para capturarlos?
  \item
    A partir del punto anterior, ¿serı́a efectiva otra combinación de
    parámetros teniendo en cuenta un nivel del pokemon más bajo (o más
    alto)?
  \end{enumerate}
\end{enumerate}

    \section{Preparando el entorno}\label{preparando-el-entorno}

Para poder realizar las consignas solicitadas primero debemos preparare
el entorno de trabajo, importando los módulos necesarios:

\begin{itemize}
\tightlist
\item
  \texttt{matplotlib}: módulo encargado de la realización de gráficos.
\item
  \texttt{numpy}: módulo que simplifica los cálculos con vectores y
  matrices.
\end{itemize}

A su vez tenemos también módulos locales:

\begin{itemize}
\tightlist
\item
  \texttt{attempt\_catch}: es la función encargada de calcular la
  probabilidad de atrapar al \textbf{Pokemon}.
\item
  \texttt{PokemonFactory}: clase encargada de generar \textbf{Pokemones}
  en base a un archivo de entrada de tipo \emph{json}.
\item
  \texttt{StatusEffect}: estructura que simula toda la información
  respecto a los posibles estados de salud del \textbf{Pokemon}.
\end{itemize}

    \begin{tcolorbox}[breakable, size=fbox, boxrule=1pt, pad at break*=1mm,colback=cellbackground, colframe=cellborder]
\prompt{In}{incolor}{1}{\boxspacing}
\begin{Verbatim}[commandchars=\\\{\}]
\PY{k+kn}{import}\PY{+w}{ }\PY{n+nn}{matplotlib}\PY{n+nn}{.}\PY{n+nn}{pyplot}\PY{+w}{ }\PY{k}{as}\PY{+w}{ }\PY{n+nn}{plt}
\PY{k+kn}{import}\PY{+w}{ }\PY{n+nn}{numpy}\PY{+w}{ }\PY{k}{as}\PY{+w}{ }\PY{n+nn}{np}

\PY{k+kn}{from}\PY{+w}{ }\PY{n+nn}{src}\PY{n+nn}{.}\PY{n+nn}{catching}\PY{+w}{ }\PY{k+kn}{import} \PY{n}{attempt\PYZus{}catch}
\PY{k+kn}{from}\PY{+w}{ }\PY{n+nn}{src}\PY{n+nn}{.}\PY{n+nn}{pokemon}\PY{+w}{ }\PY{k+kn}{import} \PY{n}{PokemonFactory}\PY{p}{,} \PY{n}{StatusEffect}
\end{Verbatim}
\end{tcolorbox}

    Ahora, creamos un \emph{factory} encargado de generar todos los
\textbf{Pokemones} en base a un archivo \emph{json} que los define.

    \begin{tcolorbox}[breakable, size=fbox, boxrule=1pt, pad at break*=1mm,colback=cellbackground, colframe=cellborder]
\prompt{In}{incolor}{2}{\boxspacing}
\begin{Verbatim}[commandchars=\\\{\}]
\PY{n}{factory} \PY{o}{=} \PY{n}{PokemonFactory}\PY{p}{(}\PY{l+s+s2}{\PYZdq{}}\PY{l+s+s2}{pokemon.json}\PY{l+s+s2}{\PYZdq{}}\PY{p}{)}
\end{Verbatim}
\end{tcolorbox}

    Definamos también los posibles \textbf{Pokemones}, junto con las
posibles \textbf{Pokeballs} a utilizar en la resolución de las
consignas.

    \begin{tcolorbox}[breakable, size=fbox, boxrule=1pt, pad at break*=1mm,colback=cellbackground, colframe=cellborder]
\prompt{In}{incolor}{3}{\boxspacing}
\begin{Verbatim}[commandchars=\\\{\}]
\PY{n}{pokemons} \PY{o}{=} \PY{p}{[}\PY{l+s+s2}{\PYZdq{}}\PY{l+s+s2}{caterpie}\PY{l+s+s2}{\PYZdq{}}\PY{p}{,} \PY{l+s+s2}{\PYZdq{}}\PY{l+s+s2}{onix}\PY{l+s+s2}{\PYZdq{}}\PY{p}{,} \PY{l+s+s2}{\PYZdq{}}\PY{l+s+s2}{jolteon}\PY{l+s+s2}{\PYZdq{}}\PY{p}{,} \PY{l+s+s2}{\PYZdq{}}\PY{l+s+s2}{snorlax}\PY{l+s+s2}{\PYZdq{}}\PY{p}{,} \PY{l+s+s2}{\PYZdq{}}\PY{l+s+s2}{mewtwo}\PY{l+s+s2}{\PYZdq{}}\PY{p}{]}
\PY{n}{pokeballs} \PY{o}{=} \PY{p}{[}\PY{l+s+s2}{\PYZdq{}}\PY{l+s+s2}{pokeball}\PY{l+s+s2}{\PYZdq{}}\PY{p}{,} \PY{l+s+s2}{\PYZdq{}}\PY{l+s+s2}{ultraball}\PY{l+s+s2}{\PYZdq{}}\PY{p}{,} \PY{l+s+s2}{\PYZdq{}}\PY{l+s+s2}{fastball}\PY{l+s+s2}{\PYZdq{}}\PY{p}{,} \PY{l+s+s2}{\PYZdq{}}\PY{l+s+s2}{heavyball}\PY{l+s+s2}{\PYZdq{}}\PY{p}{]}
\end{Verbatim}
\end{tcolorbox}

    \section{Resolución}\label{resoluciuxf3n}

\subsection{Punto 1a}\label{punto-1a}

Para resolver este punto, realizaremos 100 intentos de captura con cada
\textbf{Pokeball} por cada \textbf{Pokemon}. Calcularemos luego el
promedio de todos los \emph{capture\_rate} obtenidos.

    \begin{tcolorbox}[breakable, size=fbox, boxrule=1pt, pad at break*=1mm,colback=cellbackground, colframe=cellborder]
\prompt{In}{incolor}{4}{\boxspacing}
\begin{Verbatim}[commandchars=\\\{\}]
\PY{n}{results} \PY{o}{=} \PY{p}{\PYZob{}}\PY{p}{\PYZcb{}}

\PY{k}{for} \PY{n}{pokemon\PYZus{}name} \PY{o+ow}{in} \PY{n}{pokemons}\PY{p}{:}
    \PY{n}{results}\PY{p}{[}\PY{n}{pokemon\PYZus{}name}\PY{p}{]} \PY{o}{=} \PY{p}{\PYZob{}}\PY{p}{\PYZcb{}}
    \PY{n}{pokemon} \PY{o}{=} \PY{n}{factory}\PY{o}{.}\PY{n}{create}\PY{p}{(}\PY{n}{pokemon\PYZus{}name}\PY{p}{,} \PY{l+m+mi}{100}\PY{p}{,} \PY{n}{StatusEffect}\PY{o}{.}\PY{n}{NONE}\PY{p}{,} \PY{l+m+mi}{1}\PY{p}{)}

    \PY{k}{for} \PY{n}{ball} \PY{o+ow}{in} \PY{n}{pokeballs}\PY{p}{:}
        \PY{n}{results}\PY{p}{[}\PY{n}{pokemon\PYZus{}name}\PY{p}{]}\PY{p}{[}\PY{n}{ball}\PY{p}{]} \PY{o}{=} \PY{p}{[}\PY{p}{]}

        \PY{k}{for} \PY{n}{\PYZus{}} \PY{o+ow}{in} \PY{n+nb}{range}\PY{p}{(}\PY{l+m+mi}{100}\PY{p}{)}\PY{p}{:}
            \PY{n}{\PYZus{}}\PY{p}{,} \PY{n}{capture\PYZus{}rate} \PY{o}{=} \PY{n}{attempt\PYZus{}catch}\PY{p}{(}\PY{n}{pokemon}\PY{p}{,} \PY{n}{ball}\PY{p}{)}
            \PY{n}{results}\PY{p}{[}\PY{n}{pokemon\PYZus{}name}\PY{p}{]}\PY{p}{[}\PY{n}{ball}\PY{p}{]}\PY{o}{.}\PY{n}{append}\PY{p}{(}\PY{n}{capture\PYZus{}rate}\PY{p}{)}
\end{Verbatim}
\end{tcolorbox}

    Ahora que ya tenemos los resultados, calculemos el promedio de la
probabilidad de capturar a cada \textbf{Pokemon}. Para eso realizaremos
un gráfico por cada \textbf{Pokemon} que se intentó capturar, mostrando
el promedio de la probabilidad de captura para cada \textbf{Pokeball}.

    \begin{tcolorbox}[breakable, size=fbox, boxrule=1pt, pad at break*=1mm,colback=cellbackground, colframe=cellborder]
\prompt{In}{incolor}{5}{\boxspacing}
\begin{Verbatim}[commandchars=\\\{\}]
\PY{k}{for} \PY{n}{pokemon}\PY{p}{,} \PY{n}{result\PYZus{}data} \PY{o+ow}{in} \PY{n}{results}\PY{o}{.}\PY{n}{items}\PY{p}{(}\PY{p}{)}\PY{p}{:}
    \PY{n}{data} \PY{o}{=} \PY{p}{\PYZob{}}\PY{n}{ball}\PY{p}{:} \PY{n}{np}\PY{o}{.}\PY{n}{mean}\PY{p}{(}\PY{n}{r}\PY{p}{)} \PY{o}{*} \PY{l+m+mi}{100} \PY{k}{for} \PY{n}{ball}\PY{p}{,} \PY{n}{r} \PY{o+ow}{in} \PY{n}{result\PYZus{}data}\PY{o}{.}\PY{n}{items}\PY{p}{(}\PY{p}{)}\PY{p}{\PYZcb{}}

    \PY{n}{plt}\PY{o}{.}\PY{n}{figure}\PY{p}{(}\PY{n}{figsize}\PY{o}{=}\PY{p}{(}\PY{l+m+mi}{6}\PY{p}{,} \PY{l+m+mi}{4}\PY{p}{)}\PY{p}{)}
    \PY{n}{plt}\PY{o}{.}\PY{n}{bar}\PY{p}{(}\PY{n}{data}\PY{o}{.}\PY{n}{keys}\PY{p}{(}\PY{p}{)}\PY{p}{,} \PY{n}{data}\PY{o}{.}\PY{n}{values}\PY{p}{(}\PY{p}{)}\PY{p}{,} \PY{n}{edgecolor}\PY{o}{=}\PY{l+s+s2}{\PYZdq{}}\PY{l+s+s2}{black}\PY{l+s+s2}{\PYZdq{}}\PY{p}{,} \PY{n}{color}\PY{o}{=}\PY{l+s+s2}{\PYZdq{}}\PY{l+s+s2}{skyblue}\PY{l+s+s2}{\PYZdq{}}\PY{p}{)}

    \PY{n}{plt}\PY{o}{.}\PY{n}{xlabel}\PY{p}{(}\PY{l+s+s2}{\PYZdq{}}\PY{l+s+s2}{Tipo de Pokeball}\PY{l+s+s2}{\PYZdq{}}\PY{p}{)}
    \PY{n}{plt}\PY{o}{.}\PY{n}{ylabel}\PY{p}{(}\PY{l+s+s2}{\PYZdq{}}\PY{l+s+s2}{Probabilidad de captura}\PY{l+s+s2}{\PYZdq{}}\PY{p}{)}
    \PY{n}{plt}\PY{o}{.}\PY{n}{title}\PY{p}{(}\PY{l+s+sa}{f}\PY{l+s+s2}{\PYZdq{}}\PY{l+s+s2}{Probabilidad de captura de }\PY{l+s+si}{\PYZob{}}\PY{n}{pokemon}\PY{o}{.}\PY{n}{capitalize}\PY{p}{(}\PY{p}{)}\PY{l+s+si}{\PYZcb{}}\PY{l+s+s2}{\PYZdq{}}\PY{p}{)}

    \PY{n}{plt}\PY{o}{.}\PY{n}{grid}\PY{p}{(}\PY{n}{axis}\PY{o}{=}\PY{l+s+s2}{\PYZdq{}}\PY{l+s+s2}{y}\PY{l+s+s2}{\PYZdq{}}\PY{p}{,} \PY{n}{linestyle}\PY{o}{=}\PY{l+s+s2}{\PYZdq{}}\PY{l+s+s2}{\PYZhy{}\PYZhy{}}\PY{l+s+s2}{\PYZdq{}}\PY{p}{,} \PY{n}{alpha}\PY{o}{=}\PY{l+m+mf}{0.7}\PY{p}{)}
    \PY{n}{plt}\PY{o}{.}\PY{n}{ylim}\PY{p}{(}\PY{l+m+mi}{0}\PY{p}{,} \PY{l+m+mi}{100}\PY{p}{)}

    \PY{k}{for} \PY{n}{i}\PY{p}{,} \PY{n}{rate} \PY{o+ow}{in} \PY{n+nb}{enumerate}\PY{p}{(}\PY{n}{data}\PY{o}{.}\PY{n}{values}\PY{p}{(}\PY{p}{)}\PY{p}{)}\PY{p}{:}
        \PY{n}{plt}\PY{o}{.}\PY{n}{text}\PY{p}{(}\PY{n}{i}\PY{p}{,} \PY{n}{rate}\PY{p}{,} \PY{l+s+sa}{f}\PY{l+s+s2}{\PYZdq{}}\PY{l+s+si}{\PYZob{}}\PY{n}{rate}\PY{l+s+si}{:}\PY{l+s+s2}{.2f}\PY{l+s+si}{\PYZcb{}}\PY{l+s+s2}{\PYZpc{}}\PY{l+s+s2}{\PYZdq{}}\PY{p}{,} \PY{n}{ha}\PY{o}{=}\PY{l+s+s2}{\PYZdq{}}\PY{l+s+s2}{center}\PY{l+s+s2}{\PYZdq{}}\PY{p}{,} \PY{n}{va}\PY{o}{=}\PY{l+s+s2}{\PYZdq{}}\PY{l+s+s2}{bottom}\PY{l+s+s2}{\PYZdq{}}\PY{p}{)}

    \PY{n}{plt}\PY{o}{.}\PY{n}{tight\PYZus{}layout}\PY{p}{(}\PY{p}{)}
    \PY{n}{plt}\PY{o}{.}\PY{n}{show}\PY{p}{(}\PY{n}{block}\PY{o}{=}\PY{k+kc}{False}\PY{p}{)}
\end{Verbatim}
\end{tcolorbox}

    \begin{center}
    \adjustimage{max size={0.9\linewidth}{0.9\paperheight}}{output_10_0.png}
    \end{center}
    { \hspace*{\fill} \\}

    \begin{center}
    \adjustimage{max size={0.9\linewidth}{0.9\paperheight}}{output_10_1.png}
    \end{center}
    { \hspace*{\fill} \\}

    \begin{center}
    \adjustimage{max size={0.9\linewidth}{0.9\paperheight}}{output_10_2.png}
    \end{center}
    { \hspace*{\fill} \\}

    \begin{center}
    \adjustimage{max size={0.9\linewidth}{0.9\paperheight}}{output_10_3.png}
    \end{center}
    { \hspace*{\fill} \\}

    \begin{center}
    \adjustimage{max size={0.9\linewidth}{0.9\paperheight}}{output_10_4.png}
    \end{center}
    { \hspace*{\fill} \\}

    Podemos observar que la \textbf{Ultraball} duplica la efectividad de la
\textbf{Pokeball} básica mientras que la \textbf{Heavyball} y la
\textbf{Fastball} varían dependiendo

    \subsection{Punto 1b}\label{punto-1b}

Para resolver este punto realizamos 1000 intentos de captura con cada
\textbf{Pokeball} por cada \textbf{Pokemon}. Luego se calcula el
\emph{success rate} calculando el promedio de éxitos de cada
\textbf{Pokeball} por cada \textbf{Pokemon}. Por último por cada
\textbf{Pokeball} se calcula la efectividad relativa dividiendo su
\emph{success rate} por la de la \textbf{Pokeball} básica. Lo hacemos
con 1000 intentos porque al hacerlo con 100 nos pasaba que la
probabilidad de captura para la \textbf{Pokeball} básica quedaba muy
cercano a 0 y al calcular el \emph{success rate} relativo nos daban
números no representativos.

    \begin{tcolorbox}[breakable, size=fbox, boxrule=1pt, pad at break*=1mm,colback=cellbackground, colframe=cellborder]
\prompt{In}{incolor}{5}{\boxspacing}
\begin{Verbatim}[commandchars=\\\{\}]
\PY{n}{results} \PY{o}{=} \PY{p}{\PYZob{}}\PY{n}{pokemon}\PY{p}{:} \PY{p}{\PYZob{}}\PY{n}{ball}\PY{p}{:} \PY{l+m+mi}{0} \PY{k}{for} \PY{n}{ball} \PY{o+ow}{in} \PY{n}{pokeballs}\PY{p}{\PYZcb{}} \PY{k}{for} \PY{n}{pokemon} \PY{o+ow}{in} \PY{n}{pokemon\PYZus{}names}\PY{p}{\PYZcb{}}

\PY{k}{for} \PY{n}{pokemon\PYZus{}name} \PY{o+ow}{in} \PY{n}{pokemon\PYZus{}names}\PY{p}{:}
    \PY{k}{for} \PY{n}{ball} \PY{o+ow}{in} \PY{n}{pokeballs}\PY{p}{:}
        \PY{n}{pokemon} \PY{o}{=} \PY{n}{factory}\PY{o}{.}\PY{n}{create}\PY{p}{(}\PY{n}{pokemon\PYZus{}name}\PY{p}{,} \PY{l+m+mi}{50}\PY{p}{,} \PY{n}{StatusEffect}\PY{o}{.}\PY{n}{NONE}\PY{p}{,} \PY{l+m+mf}{1.0}\PY{p}{)}
        \PY{n}{rates} \PY{o}{=} \PY{p}{[}\PY{n}{attempt\PYZus{}catch}\PY{p}{(}\PY{n}{pokemon}\PY{p}{,} \PY{n}{ball}\PY{p}{,} \PY{l+m+mf}{0.15}\PY{p}{)}\PY{p}{[}\PY{l+m+mi}{0}\PY{p}{]} \PY{k}{for} \PY{n}{\PYZus{}} \PY{o+ow}{in} \PY{n+nb}{range}\PY{p}{(}\PY{l+m+mi}{1000}\PY{p}{)}\PY{p}{]}
        \PY{n}{results}\PY{p}{[}\PY{n}{pokemon\PYZus{}name}\PY{p}{]}\PY{p}{[}\PY{n}{ball}\PY{p}{]} \PY{o}{=} \PY{n}{np}\PY{o}{.}\PY{n}{mean}\PY{p}{(}\PY{n}{rates}\PY{p}{)}

\PY{n}{relative\PYZus{}effectiveness} \PY{o}{=} \PY{p}{\PYZob{}}\PY{p}{\PYZcb{}}

\PY{k}{for} \PY{n}{pokemon} \PY{o+ow}{in} \PY{n}{pokemon\PYZus{}names}\PY{p}{:}
    \PY{n}{pokeball\PYZus{}success} \PY{o}{=} \PY{n}{results}\PY{p}{[}\PY{n}{pokemon}\PY{p}{]}\PY{p}{[}\PY{l+s+s2}{\PYZdq{}}\PY{l+s+s2}{pokeball}\PY{l+s+s2}{\PYZdq{}}\PY{p}{]}
    \PY{n}{relative\PYZus{}effectiveness}\PY{p}{[}\PY{n}{pokemon}\PY{p}{]} \PY{o}{=} \PY{p}{\PYZob{}}\PY{n}{ball}\PY{p}{:} \PY{n}{results}\PY{p}{[}\PY{n}{pokemon}\PY{p}{]}\PY{p}{[}\PY{n}{ball}\PY{p}{]} \PY{o}{/} \PY{n}{pokeball\PYZus{}success} \PY{k}{for} \PY{n}{ball} \PY{o+ow}{in} \PY{n}{pokeballs}\PY{p}{\PYZcb{}}

\PY{n}{plt}\PY{o}{.}\PY{n}{figure}\PY{p}{(}\PY{n}{figsize}\PY{o}{=}\PY{p}{(}\PY{l+m+mi}{10}\PY{p}{,} \PY{l+m+mi}{6}\PY{p}{)}\PY{p}{)}
\PY{k}{for} \PY{n}{pokemon}\PY{p}{,} \PY{n}{effectiveness} \PY{o+ow}{in} \PY{n}{relative\PYZus{}effectiveness}\PY{o}{.}\PY{n}{items}\PY{p}{(}\PY{p}{)}\PY{p}{:}
    \PY{n}{plt}\PY{o}{.}\PY{n}{plot}\PY{p}{(}\PY{n}{pokeballs}\PY{p}{,} \PY{n+nb}{list}\PY{p}{(}\PY{n}{effectiveness}\PY{o}{.}\PY{n}{values}\PY{p}{(}\PY{p}{)}\PY{p}{)}\PY{p}{,} \PY{n}{marker}\PY{o}{=}\PY{l+s+s1}{\PYZsq{}}\PY{l+s+s1}{o}\PY{l+s+s1}{\PYZsq{}}\PY{p}{,} \PY{n}{label}\PY{o}{=}\PY{n}{pokemon}\PY{p}{)}

\PY{n}{plt}\PY{o}{.}\PY{n}{xlabel}\PY{p}{(}\PY{l+s+s2}{\PYZdq{}}\PY{l+s+s2}{Tipo de Pokébola}\PY{l+s+s2}{\PYZdq{}}\PY{p}{)}
\PY{n}{plt}\PY{o}{.}\PY{n}{ylabel}\PY{p}{(}\PY{l+s+s2}{\PYZdq{}}\PY{l+s+s2}{Efectividad Relativa}\PY{l+s+s2}{\PYZdq{}}\PY{p}{)}
\PY{n}{plt}\PY{o}{.}\PY{n}{title}\PY{p}{(}\PY{l+s+s2}{\PYZdq{}}\PY{l+s+s2}{Efectividad Relativa de las Pokébolas por Pokémon}\PY{l+s+s2}{\PYZdq{}}\PY{p}{)}
\PY{n}{plt}\PY{o}{.}\PY{n}{legend}\PY{p}{(}\PY{n}{title}\PY{o}{=}\PY{l+s+s2}{\PYZdq{}}\PY{l+s+s2}{Pokémon}\PY{l+s+s2}{\PYZdq{}}\PY{p}{)}
\PY{n}{plt}\PY{o}{.}\PY{n}{grid}\PY{p}{(}\PY{n}{axis}\PY{o}{=}\PY{l+s+s2}{\PYZdq{}}\PY{l+s+s2}{y}\PY{l+s+s2}{\PYZdq{}}\PY{p}{,} \PY{n}{linestyle}\PY{o}{=}\PY{l+s+s2}{\PYZdq{}}\PY{l+s+s2}{\PYZhy{}\PYZhy{}}\PY{l+s+s2}{\PYZdq{}}\PY{p}{,} \PY{n}{alpha}\PY{o}{=}\PY{l+m+mf}{0.7}\PY{p}{)}
\PY{n}{plt}\PY{o}{.}\PY{n}{show}\PY{p}{(}\PY{p}{)}
\end{Verbatim}
\end{tcolorbox}

    \begin{center}
    \adjustimage{max size={0.9\linewidth}{0.9\paperheight}}{output_13_0.png}
    \end{center}
    { \hspace*{\fill} \\}

    En el gráfico podemos ver claramente la efectividad de cada
\textbf{Pokebola} en relación a la \textbf{Pokebola} básica y llegamos a
la conclusión de que efectivamente es cierto que algunas
\textbf{Pokebolas} son más o menos efectivas dependiendo de las
propiedades intrínsecas de cada \textbf{Pokemon}. Por ejemplo la
\textbf{Heavyball} es más efectiva que la \textbf{Pokebola} básica para
atrapar al \textbf{Snorlax} pero menos efectiva para atrapar al
\textbf{Mewtwo}. Otro ejemplo que también se puede ver en el gráfico es
que la mejor \textbf{Pokebola} para atrapar al \textbf{Jolteon} es la
\textbf{Fastball}. Estos resultados reflejan que la elección de la
\textbf{Pokebola} adecuada puede influir significativamente en la
probabilidad de captura de cada \textbf{Pokemon}.

    \subsection{Punto 2a}\label{punto-2a}

Veamos ahora como se ve afectada la efectividad para la captura de los
\textbf{Pokemones} dependiendo de los estados de salud de los mismos.
Para eso vamos a definir los estados de salud que vamos a estar
analizando.

    \begin{tcolorbox}[breakable, size=fbox, boxrule=1pt, pad at break*=1mm,colback=cellbackground, colframe=cellborder]
\prompt{In}{incolor}{7}{\boxspacing}
\begin{Verbatim}[commandchars=\\\{\}]
\PY{n}{effects} \PY{o}{=} \PY{p}{[}\PY{n}{StatusEffect}\PY{o}{.}\PY{n}{NONE}\PY{p}{,} \PY{n}{StatusEffect}\PY{o}{.}\PY{n}{POISON}\PY{p}{,} \PY{n}{StatusEffect}\PY{o}{.}\PY{n}{BURN}\PY{p}{,} \PY{n}{StatusEffect}\PY{o}{.}\PY{n}{PARALYSIS}\PY{p}{,} \PY{n}{StatusEffect}\PY{o}{.}\PY{n}{SLEEP}\PY{p}{,} \PY{n}{StatusEffect}\PY{o}{.}\PY{n}{FREEZE}\PY{p}{]}
\end{Verbatim}
\end{tcolorbox}

    Intentemos ahora calcular a \textbf{Onix} y a \textbf{Caterpie} 100
veces cada uno en cada uno de los posibles estados de salud que
definimos.

    \begin{tcolorbox}[breakable, size=fbox, boxrule=1pt, pad at break*=1mm,colback=cellbackground, colframe=cellborder]
\prompt{In}{incolor}{8}{\boxspacing}
\begin{Verbatim}[commandchars=\\\{\}]
\PY{n}{results} \PY{o}{=} \PY{p}{\PYZob{}}\PY{p}{\PYZcb{}}

\PY{k}{for} \PY{n}{pokemon\PYZus{}name} \PY{o+ow}{in} \PY{p}{[}\PY{l+s+s2}{\PYZdq{}}\PY{l+s+s2}{onix}\PY{l+s+s2}{\PYZdq{}}\PY{p}{,} \PY{l+s+s2}{\PYZdq{}}\PY{l+s+s2}{caterpie}\PY{l+s+s2}{\PYZdq{}}\PY{p}{]}\PY{p}{:}
    \PY{n}{results}\PY{p}{[}\PY{n}{pokemon\PYZus{}name}\PY{p}{]} \PY{o}{=} \PY{p}{\PYZob{}}\PY{p}{\PYZcb{}}

    \PY{k}{for} \PY{n}{effect} \PY{o+ow}{in} \PY{n}{effects}\PY{p}{:}
        \PY{n}{results}\PY{p}{[}\PY{n}{pokemon\PYZus{}name}\PY{p}{]}\PY{p}{[}\PY{n}{effect}\PY{o}{.}\PY{n}{name}\PY{p}{]} \PY{o}{=} \PY{p}{[}\PY{p}{]}
        \PY{n}{pokemon} \PY{o}{=} \PY{n}{factory}\PY{o}{.}\PY{n}{create}\PY{p}{(}\PY{n}{pokemon\PYZus{}name}\PY{p}{,} \PY{l+m+mi}{100}\PY{p}{,} \PY{n}{effect}\PY{p}{,} \PY{l+m+mi}{1}\PY{p}{)}

        \PY{k}{for} \PY{n}{\PYZus{}} \PY{o+ow}{in} \PY{n+nb}{range}\PY{p}{(}\PY{l+m+mi}{100}\PY{p}{)}\PY{p}{:}
            \PY{n}{success}\PY{p}{,} \PY{n}{\PYZus{}} \PY{o}{=} \PY{n}{attempt\PYZus{}catch}\PY{p}{(}\PY{n}{pokemon}\PY{p}{,} \PY{l+s+s2}{\PYZdq{}}\PY{l+s+s2}{pokeball}\PY{l+s+s2}{\PYZdq{}}\PY{p}{,} \PY{l+m+mf}{0.15}\PY{p}{)}
            \PY{n}{results}\PY{p}{[}\PY{n}{pokemon\PYZus{}name}\PY{p}{]}\PY{p}{[}\PY{n}{effect}\PY{o}{.}\PY{n}{name}\PY{p}{]}\PY{o}{.}\PY{n}{append}\PY{p}{(}\PY{n}{success}\PY{p}{)}
\end{Verbatim}
\end{tcolorbox}

    Ahora grafiquemos para ver como varía la efectividad de los diferentes
estados de salud.

    \begin{tcolorbox}[breakable, size=fbox, boxrule=1pt, pad at break*=1mm,colback=cellbackground, colframe=cellborder]
\prompt{In}{incolor}{9}{\boxspacing}
\begin{Verbatim}[commandchars=\\\{\}]
\PY{k}{for} \PY{n}{pokemon}\PY{p}{,} \PY{n}{result\PYZus{}data} \PY{o+ow}{in} \PY{n}{results}\PY{o}{.}\PY{n}{items}\PY{p}{(}\PY{p}{)}\PY{p}{:}
    \PY{n}{data} \PY{o}{=} \PY{p}{\PYZob{}}\PY{n}{effect}\PY{p}{:} \PY{n}{np}\PY{o}{.}\PY{n}{mean}\PY{p}{(}\PY{n}{r}\PY{p}{)} \PY{o}{*} \PY{l+m+mi}{100} \PY{k}{for} \PY{n}{effect}\PY{p}{,} \PY{n}{r} \PY{o+ow}{in} \PY{n}{result\PYZus{}data}\PY{o}{.}\PY{n}{items}\PY{p}{(}\PY{p}{)}\PY{p}{\PYZcb{}}

    \PY{n}{plt}\PY{o}{.}\PY{n}{figure}\PY{p}{(}\PY{n}{figsize}\PY{o}{=}\PY{p}{(}\PY{l+m+mi}{6}\PY{p}{,} \PY{l+m+mi}{4}\PY{p}{)}\PY{p}{)}
    \PY{n}{plt}\PY{o}{.}\PY{n}{bar}\PY{p}{(}\PY{n}{data}\PY{o}{.}\PY{n}{keys}\PY{p}{(}\PY{p}{)}\PY{p}{,} \PY{n}{data}\PY{o}{.}\PY{n}{values}\PY{p}{(}\PY{p}{)}\PY{p}{,} \PY{n}{edgecolor}\PY{o}{=}\PY{l+s+s2}{\PYZdq{}}\PY{l+s+s2}{black}\PY{l+s+s2}{\PYZdq{}}\PY{p}{,} \PY{n}{color}\PY{o}{=}\PY{l+s+s2}{\PYZdq{}}\PY{l+s+s2}{skyblue}\PY{l+s+s2}{\PYZdq{}}\PY{p}{)}

    \PY{n}{plt}\PY{o}{.}\PY{n}{xlabel}\PY{p}{(}\PY{l+s+s2}{\PYZdq{}}\PY{l+s+s2}{Estado de Salud}\PY{l+s+s2}{\PYZdq{}}\PY{p}{)}
    \PY{n}{plt}\PY{o}{.}\PY{n}{ylabel}\PY{p}{(}\PY{l+s+s2}{\PYZdq{}}\PY{l+s+s2}{Probabilidad de Captura}\PY{l+s+s2}{\PYZdq{}}\PY{p}{)}
    \PY{n}{plt}\PY{o}{.}\PY{n}{title}\PY{p}{(}\PY{l+s+sa}{f}\PY{l+s+s2}{\PYZdq{}}\PY{l+s+s2}{Efecto del Estado de Salud en la Captura de }\PY{l+s+si}{\PYZob{}}\PY{n}{pokemon}\PY{o}{.}\PY{n}{capitalize}\PY{p}{(}\PY{p}{)}\PY{l+s+si}{\PYZcb{}}\PY{l+s+s2}{\PYZdq{}}\PY{p}{)}

    \PY{n}{plt}\PY{o}{.}\PY{n}{grid}\PY{p}{(}\PY{n}{axis}\PY{o}{=}\PY{l+s+s2}{\PYZdq{}}\PY{l+s+s2}{y}\PY{l+s+s2}{\PYZdq{}}\PY{p}{,} \PY{n}{linestyle}\PY{o}{=}\PY{l+s+s2}{\PYZdq{}}\PY{l+s+s2}{\PYZhy{}\PYZhy{}}\PY{l+s+s2}{\PYZdq{}}\PY{p}{,} \PY{n}{alpha}\PY{o}{=}\PY{l+m+mf}{0.7}\PY{p}{)}
    \PY{n}{plt}\PY{o}{.}\PY{n}{ylim}\PY{p}{(}\PY{l+m+mi}{0}\PY{p}{,} \PY{l+m+mi}{110}\PY{p}{)}

    \PY{k}{for} \PY{n}{i}\PY{p}{,} \PY{n}{rate} \PY{o+ow}{in} \PY{n+nb}{enumerate}\PY{p}{(}\PY{n}{data}\PY{o}{.}\PY{n}{values}\PY{p}{(}\PY{p}{)}\PY{p}{)}\PY{p}{:}
        \PY{n}{plt}\PY{o}{.}\PY{n}{text}\PY{p}{(}\PY{n}{i}\PY{p}{,} \PY{n}{rate}\PY{p}{,} \PY{l+s+sa}{f}\PY{l+s+s2}{\PYZdq{}}\PY{l+s+si}{\PYZob{}}\PY{n}{rate}\PY{l+s+si}{:}\PY{l+s+s2}{.2f}\PY{l+s+si}{\PYZcb{}}\PY{l+s+s2}{\PYZpc{}}\PY{l+s+s2}{\PYZdq{}}\PY{p}{,} \PY{n}{ha}\PY{o}{=}\PY{l+s+s2}{\PYZdq{}}\PY{l+s+s2}{center}\PY{l+s+s2}{\PYZdq{}}\PY{p}{,} \PY{n}{va}\PY{o}{=}\PY{l+s+s2}{\PYZdq{}}\PY{l+s+s2}{bottom}\PY{l+s+s2}{\PYZdq{}}\PY{p}{)}

    \PY{n}{plt}\PY{o}{.}\PY{n}{tight\PYZus{}layout}\PY{p}{(}\PY{p}{)}
    \PY{n}{plt}\PY{o}{.}\PY{n}{show}\PY{p}{(}\PY{p}{)}
\end{Verbatim}
\end{tcolorbox}

    \begin{center}
    \adjustimage{max size={0.9\linewidth}{0.9\paperheight}}{output_20_0.png}
    \end{center}
    { \hspace*{\fill} \\}

    \begin{center}
    \adjustimage{max size={0.9\linewidth}{0.9\paperheight}}{output_20_1.png}
    \end{center}
    { \hspace*{\fill} \\}

    Por más que el porcentaje de efectividad en cada caso es diferente,
podemos ver claramente como el estado de ``\emph{sueño}'' y
``\emph{congelamiento}'' afectan en mucho mayor medida a la efectividad
en la captura de los \textbf{Pokemones}. Incluso en menores grados,
podemos ver que todos los efectos de salud, aumentan a la efectividad de
la captura, ya que en comparación al caso de que no se haya aplicado
ningún efecto, estos son mucho mayores.

    \subsection{Punto 2b}\label{punto-2b}

Veamos ahora, de forma similar, como afectan los puntos de vida
restantes al intentar capturar a los \textbf{Pokemones}. Para este caso,
usaremos los mismos \textbf{Pokemones} del caso anterior (\textbf{Onix}
y \textbf{Caterpie}), manteniendo el nivel en 100, sin efectos de salud
y utilizando la \textbf{Pokeball} básica.

Definamos entonces para eso los valores de vida que estaremos
utilizando.

    \begin{tcolorbox}[breakable, size=fbox, boxrule=1pt, pad at break*=1mm,colback=cellbackground, colframe=cellborder]
\prompt{In}{incolor}{10}{\boxspacing}
\begin{Verbatim}[commandchars=\\\{\}]
\PY{n}{healths} \PY{o}{=} \PY{p}{[}\PY{l+s+s2}{\PYZdq{}}\PY{l+s+s2}{1.00}\PY{l+s+s2}{\PYZdq{}}\PY{p}{,} \PY{l+s+s2}{\PYZdq{}}\PY{l+s+s2}{0.75}\PY{l+s+s2}{\PYZdq{}}\PY{p}{,} \PY{l+s+s2}{\PYZdq{}}\PY{l+s+s2}{0.50}\PY{l+s+s2}{\PYZdq{}}\PY{p}{,} \PY{l+s+s2}{\PYZdq{}}\PY{l+s+s2}{0.25}\PY{l+s+s2}{\PYZdq{}}\PY{p}{,} \PY{l+s+s2}{\PYZdq{}}\PY{l+s+s2}{0.01}\PY{l+s+s2}{\PYZdq{}}\PY{p}{]}
\end{Verbatim}
\end{tcolorbox}

    Ahora, intentemos atrapar a los \textbf{Pokemones} mencionados, 100
veces cada uno por cada puntaje de vida disponible.

    \begin{tcolorbox}[breakable, size=fbox, boxrule=1pt, pad at break*=1mm,colback=cellbackground, colframe=cellborder]
\prompt{In}{incolor}{11}{\boxspacing}
\begin{Verbatim}[commandchars=\\\{\}]
\PY{n}{results} \PY{o}{=} \PY{p}{\PYZob{}}\PY{p}{\PYZcb{}}

\PY{k}{for} \PY{n}{pokemon\PYZus{}name} \PY{o+ow}{in} \PY{p}{[}\PY{l+s+s2}{\PYZdq{}}\PY{l+s+s2}{onix}\PY{l+s+s2}{\PYZdq{}}\PY{p}{,} \PY{l+s+s2}{\PYZdq{}}\PY{l+s+s2}{caterpie}\PY{l+s+s2}{\PYZdq{}}\PY{p}{]}\PY{p}{:}
    \PY{n}{results}\PY{p}{[}\PY{n}{pokemon\PYZus{}name}\PY{p}{]} \PY{o}{=} \PY{p}{\PYZob{}}\PY{p}{\PYZcb{}}

    \PY{k}{for} \PY{n}{health} \PY{o+ow}{in} \PY{n}{healths}\PY{p}{:}
        \PY{n}{results}\PY{p}{[}\PY{n}{pokemon\PYZus{}name}\PY{p}{]}\PY{p}{[}\PY{n}{health}\PY{p}{]} \PY{o}{=} \PY{p}{[}\PY{p}{]}
        \PY{n}{pokemon} \PY{o}{=} \PY{n}{factory}\PY{o}{.}\PY{n}{create}\PY{p}{(}\PY{n}{pokemon\PYZus{}name}\PY{p}{,} \PY{l+m+mi}{100}\PY{p}{,} \PY{n}{StatusEffect}\PY{o}{.}\PY{n}{NONE}\PY{p}{,} \PY{n+nb}{float}\PY{p}{(}\PY{n}{health}\PY{p}{)}\PY{p}{)}

        \PY{k}{for} \PY{n}{\PYZus{}} \PY{o+ow}{in} \PY{n+nb}{range}\PY{p}{(}\PY{l+m+mi}{100}\PY{p}{)}\PY{p}{:}
            \PY{n}{success}\PY{p}{,} \PY{n}{\PYZus{}} \PY{o}{=} \PY{n}{attempt\PYZus{}catch}\PY{p}{(}\PY{n}{pokemon}\PY{p}{,} \PY{l+s+s2}{\PYZdq{}}\PY{l+s+s2}{pokeball}\PY{l+s+s2}{\PYZdq{}}\PY{p}{)}
            \PY{n}{results}\PY{p}{[}\PY{n}{pokemon\PYZus{}name}\PY{p}{]}\PY{p}{[}\PY{n}{health}\PY{p}{]}\PY{o}{.}\PY{n}{append}\PY{p}{(}\PY{n}{success}\PY{p}{)}
\end{Verbatim}
\end{tcolorbox}

    Con los resultados disponibles, calculemos la efectividad en cada caso y
grafiquemos los resultados.

    \begin{tcolorbox}[breakable, size=fbox, boxrule=1pt, pad at break*=1mm,colback=cellbackground, colframe=cellborder]
\prompt{In}{incolor}{12}{\boxspacing}
\begin{Verbatim}[commandchars=\\\{\}]
\PY{k}{for} \PY{n}{pokemon}\PY{p}{,} \PY{n}{result\PYZus{}data} \PY{o+ow}{in} \PY{n}{results}\PY{o}{.}\PY{n}{items}\PY{p}{(}\PY{p}{)}\PY{p}{:}
    \PY{n}{data} \PY{o}{=} \PY{p}{\PYZob{}}\PY{n}{health}\PY{p}{:} \PY{n}{np}\PY{o}{.}\PY{n}{mean}\PY{p}{(}\PY{n}{r}\PY{p}{)} \PY{o}{*} \PY{l+m+mi}{100} \PY{k}{for} \PY{n}{health}\PY{p}{,} \PY{n}{r} \PY{o+ow}{in} \PY{n}{result\PYZus{}data}\PY{o}{.}\PY{n}{items}\PY{p}{(}\PY{p}{)}\PY{p}{\PYZcb{}}

    \PY{n}{plt}\PY{o}{.}\PY{n}{figure}\PY{p}{(}\PY{n}{figsize}\PY{o}{=}\PY{p}{(}\PY{l+m+mi}{6}\PY{p}{,} \PY{l+m+mi}{4}\PY{p}{)}\PY{p}{)}
    \PY{n}{plt}\PY{o}{.}\PY{n}{bar}\PY{p}{(}\PY{n}{data}\PY{o}{.}\PY{n}{keys}\PY{p}{(}\PY{p}{)}\PY{p}{,} \PY{n}{data}\PY{o}{.}\PY{n}{values}\PY{p}{(}\PY{p}{)}\PY{p}{,} \PY{n}{edgecolor}\PY{o}{=}\PY{l+s+s2}{\PYZdq{}}\PY{l+s+s2}{black}\PY{l+s+s2}{\PYZdq{}}\PY{p}{,} \PY{n}{color}\PY{o}{=}\PY{l+s+s2}{\PYZdq{}}\PY{l+s+s2}{skyblue}\PY{l+s+s2}{\PYZdq{}}\PY{p}{)}

    \PY{n}{plt}\PY{o}{.}\PY{n}{xlabel}\PY{p}{(}\PY{l+s+s2}{\PYZdq{}}\PY{l+s+s2}{Porcentaje de vida}\PY{l+s+s2}{\PYZdq{}}\PY{p}{)}
    \PY{n}{plt}\PY{o}{.}\PY{n}{ylabel}\PY{p}{(}\PY{l+s+s2}{\PYZdq{}}\PY{l+s+s2}{Probabilidad de captura}\PY{l+s+s2}{\PYZdq{}}\PY{p}{)}
    \PY{n}{plt}\PY{o}{.}\PY{n}{title}\PY{p}{(}\PY{l+s+sa}{f}\PY{l+s+s2}{\PYZdq{}}\PY{l+s+s2}{Efecto de la Vida en la Captura de }\PY{l+s+si}{\PYZob{}}\PY{n}{pokemon}\PY{o}{.}\PY{n}{capitalize}\PY{p}{(}\PY{p}{)}\PY{l+s+si}{\PYZcb{}}\PY{l+s+s2}{\PYZdq{}}\PY{p}{)}

    \PY{n}{plt}\PY{o}{.}\PY{n}{grid}\PY{p}{(}\PY{n}{axis}\PY{o}{=}\PY{l+s+s2}{\PYZdq{}}\PY{l+s+s2}{y}\PY{l+s+s2}{\PYZdq{}}\PY{p}{,} \PY{n}{linestyle}\PY{o}{=}\PY{l+s+s2}{\PYZdq{}}\PY{l+s+s2}{\PYZhy{}\PYZhy{}}\PY{l+s+s2}{\PYZdq{}}\PY{p}{,} \PY{n}{alpha}\PY{o}{=}\PY{l+m+mf}{0.7}\PY{p}{)}
    \PY{n}{plt}\PY{o}{.}\PY{n}{ylim}\PY{p}{(}\PY{l+m+mi}{0}\PY{p}{,} \PY{l+m+mi}{110}\PY{p}{)}

    \PY{k}{for} \PY{n}{i}\PY{p}{,} \PY{n}{rate} \PY{o+ow}{in} \PY{n+nb}{enumerate}\PY{p}{(}\PY{n}{data}\PY{o}{.}\PY{n}{values}\PY{p}{(}\PY{p}{)}\PY{p}{)}\PY{p}{:}
        \PY{n}{plt}\PY{o}{.}\PY{n}{text}\PY{p}{(}\PY{n}{i}\PY{p}{,} \PY{n}{rate}\PY{p}{,} \PY{l+s+sa}{f}\PY{l+s+s2}{\PYZdq{}}\PY{l+s+si}{\PYZob{}}\PY{n}{rate}\PY{l+s+si}{:}\PY{l+s+s2}{.2f}\PY{l+s+si}{\PYZcb{}}\PY{l+s+s2}{\PYZpc{}}\PY{l+s+s2}{\PYZdq{}}\PY{p}{,} \PY{n}{ha}\PY{o}{=}\PY{l+s+s2}{\PYZdq{}}\PY{l+s+s2}{center}\PY{l+s+s2}{\PYZdq{}}\PY{p}{,} \PY{n}{va}\PY{o}{=}\PY{l+s+s2}{\PYZdq{}}\PY{l+s+s2}{bottom}\PY{l+s+s2}{\PYZdq{}}\PY{p}{)}

    \PY{n}{plt}\PY{o}{.}\PY{n}{tight\PYZus{}layout}\PY{p}{(}\PY{p}{)}
    \PY{n}{plt}\PY{o}{.}\PY{n}{show}\PY{p}{(}\PY{p}{)}
\end{Verbatim}
\end{tcolorbox}

    \begin{center}
    \adjustimage{max size={0.9\linewidth}{0.9\paperheight}}{output_27_0.png}
    \end{center}
    { \hspace*{\fill} \\}

    \begin{center}
    \adjustimage{max size={0.9\linewidth}{0.9\paperheight}}{output_27_1.png}
    \end{center}
    { \hspace*{\fill} \\}

    Se puede observar que cuando el nivel de vida es muy bajo, aumenta
considerablemente la probabilidad de captura.

    \subsection{Punto 2c}\label{punto-2c}

Ya analizamos como varía la efectividad al agregar efectos de salud y al
modificar los puntos de vida del \textbf{Pokemon}. Veamos por último
como varía la efectividad al modificar el nivel de \textbf{Pokemon}.
Nuevamente, probaremos con \textbf{Onix} y \textbf{Caterpie}, que son
los que estuvimos utilizando hasta el momento. Definamos entonces los
posibles niveles a probar.

    \begin{tcolorbox}[breakable, size=fbox, boxrule=1pt, pad at break*=1mm,colback=cellbackground, colframe=cellborder]
\prompt{In}{incolor}{14}{\boxspacing}
\begin{Verbatim}[commandchars=\\\{\}]
\PY{n}{levels} \PY{o}{=} \PY{p}{[}\PY{l+s+s2}{\PYZdq{}}\PY{l+s+s2}{1}\PY{l+s+s2}{\PYZdq{}}\PY{p}{,} \PY{l+s+s2}{\PYZdq{}}\PY{l+s+s2}{25}\PY{l+s+s2}{\PYZdq{}}\PY{p}{,} \PY{l+s+s2}{\PYZdq{}}\PY{l+s+s2}{50}\PY{l+s+s2}{\PYZdq{}}\PY{p}{,} \PY{l+s+s2}{\PYZdq{}}\PY{l+s+s2}{75}\PY{l+s+s2}{\PYZdq{}}\PY{p}{,} \PY{l+s+s2}{\PYZdq{}}\PY{l+s+s2}{100}\PY{l+s+s2}{\PYZdq{}}\PY{p}{]}
\end{Verbatim}
\end{tcolorbox}

    Manteniendo la vida al máximo y sin aplicar efectos de salud, veamos
como varía la efectividad al intentar atrapar a los \textbf{Pokemones}
mencionados 100 veces.

    \begin{tcolorbox}[breakable, size=fbox, boxrule=1pt, pad at break*=1mm,colback=cellbackground, colframe=cellborder]
\prompt{In}{incolor}{15}{\boxspacing}
\begin{Verbatim}[commandchars=\\\{\}]
\PY{n}{results} \PY{o}{=} \PY{p}{\PYZob{}}\PY{p}{\PYZcb{}}

\PY{k}{for} \PY{n}{pokemon\PYZus{}name} \PY{o+ow}{in} \PY{p}{[}\PY{l+s+s2}{\PYZdq{}}\PY{l+s+s2}{onix}\PY{l+s+s2}{\PYZdq{}}\PY{p}{,} \PY{l+s+s2}{\PYZdq{}}\PY{l+s+s2}{caterpie}\PY{l+s+s2}{\PYZdq{}}\PY{p}{]}\PY{p}{:}
    \PY{n}{results}\PY{p}{[}\PY{n}{pokemon\PYZus{}name}\PY{p}{]} \PY{o}{=} \PY{p}{\PYZob{}}\PY{p}{\PYZcb{}}

    \PY{k}{for} \PY{n}{level} \PY{o+ow}{in} \PY{n}{levels}\PY{p}{:}
        \PY{n}{results}\PY{p}{[}\PY{n}{pokemon\PYZus{}name}\PY{p}{]}\PY{p}{[}\PY{n}{level}\PY{p}{]} \PY{o}{=} \PY{p}{[}\PY{p}{]}
        \PY{n}{pokemon} \PY{o}{=} \PY{n}{factory}\PY{o}{.}\PY{n}{create}\PY{p}{(}\PY{n}{pokemon\PYZus{}name}\PY{p}{,} \PY{n+nb}{int}\PY{p}{(}\PY{n}{level}\PY{p}{)}\PY{p}{,} \PY{n}{StatusEffect}\PY{o}{.}\PY{n}{NONE}\PY{p}{,} \PY{l+m+mi}{1}\PY{p}{)}

        \PY{k}{for} \PY{n}{\PYZus{}} \PY{o+ow}{in} \PY{n+nb}{range}\PY{p}{(}\PY{l+m+mi}{100}\PY{p}{)}\PY{p}{:}
            \PY{n}{success}\PY{p}{,} \PY{n}{\PYZus{}} \PY{o}{=} \PY{n}{attempt\PYZus{}catch}\PY{p}{(}\PY{n}{pokemon}\PY{p}{,} \PY{l+s+s2}{\PYZdq{}}\PY{l+s+s2}{pokeball}\PY{l+s+s2}{\PYZdq{}}\PY{p}{)}
            \PY{n}{results}\PY{p}{[}\PY{n}{pokemon\PYZus{}name}\PY{p}{]}\PY{p}{[}\PY{n}{level}\PY{p}{]}\PY{o}{.}\PY{n}{append}\PY{p}{(}\PY{n}{success}\PY{p}{)}
\end{Verbatim}
\end{tcolorbox}

    Con los resultados disponibles, calculemos la efectividad por cada nivel
y grafiquemos los valores obtenidos.

    \begin{tcolorbox}[breakable, size=fbox, boxrule=1pt, pad at break*=1mm,colback=cellbackground, colframe=cellborder]
\prompt{In}{incolor}{16}{\boxspacing}
\begin{Verbatim}[commandchars=\\\{\}]
\PY{k}{for} \PY{n}{pokemon}\PY{p}{,} \PY{n}{result\PYZus{}data} \PY{o+ow}{in} \PY{n}{results}\PY{o}{.}\PY{n}{items}\PY{p}{(}\PY{p}{)}\PY{p}{:}
    \PY{n}{data} \PY{o}{=} \PY{p}{\PYZob{}}\PY{n}{level}\PY{p}{:} \PY{n}{np}\PY{o}{.}\PY{n}{mean}\PY{p}{(}\PY{n}{r}\PY{p}{)} \PY{o}{*} \PY{l+m+mi}{100} \PY{k}{for} \PY{n}{level}\PY{p}{,} \PY{n}{r} \PY{o+ow}{in} \PY{n}{result\PYZus{}data}\PY{o}{.}\PY{n}{items}\PY{p}{(}\PY{p}{)}\PY{p}{\PYZcb{}}

    \PY{n}{plt}\PY{o}{.}\PY{n}{figure}\PY{p}{(}\PY{n}{figsize}\PY{o}{=}\PY{p}{(}\PY{l+m+mi}{6}\PY{p}{,} \PY{l+m+mi}{4}\PY{p}{)}\PY{p}{)}
    \PY{n}{plt}\PY{o}{.}\PY{n}{bar}\PY{p}{(}\PY{n}{data}\PY{o}{.}\PY{n}{keys}\PY{p}{(}\PY{p}{)}\PY{p}{,} \PY{n}{data}\PY{o}{.}\PY{n}{values}\PY{p}{(}\PY{p}{)}\PY{p}{,} \PY{n}{edgecolor}\PY{o}{=}\PY{l+s+s2}{\PYZdq{}}\PY{l+s+s2}{black}\PY{l+s+s2}{\PYZdq{}}\PY{p}{,} \PY{n}{color}\PY{o}{=}\PY{l+s+s2}{\PYZdq{}}\PY{l+s+s2}{skyblue}\PY{l+s+s2}{\PYZdq{}}\PY{p}{)}

    \PY{n}{plt}\PY{o}{.}\PY{n}{xlabel}\PY{p}{(}\PY{l+s+s2}{\PYZdq{}}\PY{l+s+s2}{Nivel del Pokemon}\PY{l+s+s2}{\PYZdq{}}\PY{p}{)}
    \PY{n}{plt}\PY{o}{.}\PY{n}{ylabel}\PY{p}{(}\PY{l+s+s2}{\PYZdq{}}\PY{l+s+s2}{Probabilidad de captura}\PY{l+s+s2}{\PYZdq{}}\PY{p}{)}
    \PY{n}{plt}\PY{o}{.}\PY{n}{title}\PY{p}{(}\PY{l+s+sa}{f}\PY{l+s+s2}{\PYZdq{}}\PY{l+s+s2}{Efecto del Nivel en la Captura de }\PY{l+s+si}{\PYZob{}}\PY{n}{pokemon}\PY{o}{.}\PY{n}{capitalize}\PY{p}{(}\PY{p}{)}\PY{l+s+si}{\PYZcb{}}\PY{l+s+s2}{\PYZdq{}}\PY{p}{)}

    \PY{n}{plt}\PY{o}{.}\PY{n}{grid}\PY{p}{(}\PY{n}{axis}\PY{o}{=}\PY{l+s+s2}{\PYZdq{}}\PY{l+s+s2}{y}\PY{l+s+s2}{\PYZdq{}}\PY{p}{,} \PY{n}{linestyle}\PY{o}{=}\PY{l+s+s2}{\PYZdq{}}\PY{l+s+s2}{\PYZhy{}\PYZhy{}}\PY{l+s+s2}{\PYZdq{}}\PY{p}{,} \PY{n}{alpha}\PY{o}{=}\PY{l+m+mf}{0.7}\PY{p}{)}
    \PY{n}{plt}\PY{o}{.}\PY{n}{ylim}\PY{p}{(}\PY{l+m+mi}{0}\PY{p}{,} \PY{l+m+mi}{100}\PY{p}{)}

    \PY{k}{for} \PY{n}{i}\PY{p}{,} \PY{n}{rate} \PY{o+ow}{in} \PY{n+nb}{enumerate}\PY{p}{(}\PY{n}{data}\PY{o}{.}\PY{n}{values}\PY{p}{(}\PY{p}{)}\PY{p}{)}\PY{p}{:}
        \PY{n}{plt}\PY{o}{.}\PY{n}{text}\PY{p}{(}\PY{n}{i}\PY{p}{,} \PY{n}{rate}\PY{p}{,} \PY{l+s+sa}{f}\PY{l+s+s2}{\PYZdq{}}\PY{l+s+si}{\PYZob{}}\PY{n}{rate}\PY{l+s+si}{:}\PY{l+s+s2}{.2f}\PY{l+s+si}{\PYZcb{}}\PY{l+s+s2}{\PYZpc{}}\PY{l+s+s2}{\PYZdq{}}\PY{p}{,} \PY{n}{ha}\PY{o}{=}\PY{l+s+s2}{\PYZdq{}}\PY{l+s+s2}{center}\PY{l+s+s2}{\PYZdq{}}\PY{p}{,} \PY{n}{va}\PY{o}{=}\PY{l+s+s2}{\PYZdq{}}\PY{l+s+s2}{bottom}\PY{l+s+s2}{\PYZdq{}}\PY{p}{)}

    \PY{n}{plt}\PY{o}{.}\PY{n}{tight\PYZus{}layout}\PY{p}{(}\PY{p}{)}
    \PY{n}{plt}\PY{o}{.}\PY{n}{show}\PY{p}{(}\PY{p}{)}
\end{Verbatim}
\end{tcolorbox}

    \begin{center}
    \adjustimage{max size={0.9\linewidth}{0.9\paperheight}}{output_34_0.png}
    \end{center}
    { \hspace*{\fill} \\}

    \begin{center}
    \adjustimage{max size={0.9\linewidth}{0.9\paperheight}}{output_34_1.png}
    \end{center}
    { \hspace*{\fill} \\}

    \begin{tcolorbox}[breakable, size=fbox, boxrule=1pt, pad at break*=1mm,colback=cellbackground, colframe=cellborder]
\prompt{In}{incolor}{17}{\boxspacing}
\begin{Verbatim}[commandchars=\\\{\}]
\PY{n}{El} \PY{n}{parametro} \PY{n}{que} \PY{n}{mas} \PY{n}{afecta} \PY{n}{la} \PY{n}{probabilidad} \PY{n}{de} \PY{n}{captura} \PY{n}{es} \PY{n}{el} \PY{n}{nivel} \PY{n}{de} \PY{n}{vida} \PY{n}{de} \PY{n}{un} \PY{n}{pokemon} \PY{n}{seguido} \PY{n}{por} \PY{n}{su} \PY{n}{estado} \PY{n}{de} \PY{n}{salud} \PY{n}{y} \PY{n}{por} \PY{n}{ultimo} \PY{n}{su} \PY{n}{nivel}\PY{o}{.}
\end{Verbatim}
\end{tcolorbox}

    \subsection{Punto 2d}\label{punto-2d}

Ya analizamos como afectan a la efectividad la variación de efectos de
salud, puntos de vida restantes y el nivel del \textbf{Pokemon}. Veamos
ahora como la combinación de diferentes valores de puntos de vida,
estados de salud y \textbf{Pokeballs} diferentes.

    \begin{tcolorbox}[breakable, size=fbox, boxrule=1pt, pad at break*=1mm,colback=cellbackground, colframe=cellborder]
\prompt{In}{incolor}{18}{\boxspacing}
\begin{Verbatim}[commandchars=\\\{\}]
\PY{n}{results} \PY{o}{=} \PY{p}{\PYZob{}}\PY{p}{\PYZcb{}}

\PY{k}{for} \PY{n}{pokemon\PYZus{}name} \PY{o+ow}{in} \PY{p}{[}\PY{l+s+s2}{\PYZdq{}}\PY{l+s+s2}{onix}\PY{l+s+s2}{\PYZdq{}}\PY{p}{,} \PY{l+s+s2}{\PYZdq{}}\PY{l+s+s2}{mewtwo}\PY{l+s+s2}{\PYZdq{}}\PY{p}{]}\PY{p}{:}
    \PY{n}{results}\PY{p}{[}\PY{n}{pokemon\PYZus{}name}\PY{p}{]} \PY{o}{=} \PY{p}{[}\PY{p}{]}

    \PY{k}{for} \PY{n}{effect} \PY{o+ow}{in} \PY{n}{effects}\PY{p}{:}
        \PY{k}{for} \PY{n}{health} \PY{o+ow}{in} \PY{n}{healths}\PY{p}{:}
            \PY{k}{for} \PY{n}{ball} \PY{o+ow}{in} \PY{n}{pokeballs}\PY{p}{:}
                \PY{n}{pokemon} \PY{o}{=} \PY{n}{factory}\PY{o}{.}\PY{n}{create}\PY{p}{(}\PY{n}{pokemon\PYZus{}name}\PY{p}{,} \PY{l+m+mi}{100}\PY{p}{,} \PY{n}{effect}\PY{p}{,} \PY{n+nb}{float}\PY{p}{(}\PY{n}{health}\PY{p}{)}\PY{p}{)}
                \PY{n}{probs} \PY{o}{=} \PY{p}{[}\PY{p}{]}

                \PY{k}{for} \PY{n}{\PYZus{}} \PY{o+ow}{in} \PY{n+nb}{range}\PY{p}{(}\PY{l+m+mi}{100}\PY{p}{)}\PY{p}{:}
                    \PY{n}{success}\PY{p}{,} \PY{n}{\PYZus{}} \PY{o}{=} \PY{n}{attempt\PYZus{}catch}\PY{p}{(}\PY{n}{pokemon}\PY{p}{,} \PY{n}{ball}\PY{p}{)}
                    \PY{n}{probs}\PY{o}{.}\PY{n}{append}\PY{p}{(}\PY{n}{success}\PY{p}{)}

                \PY{n}{results}\PY{p}{[}\PY{n}{pokemon\PYZus{}name}\PY{p}{]}\PY{o}{.}\PY{n}{append}\PY{p}{(}\PY{p}{(}\PY{n}{effect}\PY{o}{.}\PY{n}{name}\PY{p}{,} \PY{n}{health}\PY{p}{,} \PY{n}{ball}\PY{p}{,} \PY{n}{np}\PY{o}{.}\PY{n}{mean}\PY{p}{(}\PY{n}{probs}\PY{p}{)} \PY{o}{*} \PY{l+m+mi}{100}\PY{p}{)}\PY{p}{)}
\end{Verbatim}
\end{tcolorbox}

    \begin{tcolorbox}[breakable, size=fbox, boxrule=1pt, pad at break*=1mm,colback=cellbackground, colframe=cellborder]
\prompt{In}{incolor}{19}{\boxspacing}
\begin{Verbatim}[commandchars=\\\{\}]
\PY{k}{for} \PY{n}{pokemon}\PY{p}{,} \PY{n}{result\PYZus{}data} \PY{o+ow}{in} \PY{n}{results}\PY{o}{.}\PY{n}{items}\PY{p}{(}\PY{p}{)}\PY{p}{:}
    \PY{n}{data} \PY{o}{=} \PY{n+nb}{sorted}\PY{p}{(}\PY{n}{result\PYZus{}data}\PY{p}{,} \PY{n}{key}\PY{o}{=}\PY{k}{lambda} \PY{n}{x}\PY{p}{:} \PY{n}{x}\PY{p}{[}\PY{l+m+mi}{3}\PY{p}{]}\PY{p}{)}\PY{p}{[}\PY{o}{\PYZhy{}}\PY{l+m+mi}{10}\PY{p}{:}\PY{p}{]}
    \PY{n}{labels} \PY{o}{=} \PY{p}{[}\PY{l+s+sa}{f}\PY{l+s+s2}{\PYZdq{}}\PY{l+s+si}{\PYZob{}}\PY{n}{x}\PY{p}{[}\PY{l+m+mi}{0}\PY{p}{]}\PY{l+s+si}{\PYZcb{}}\PY{l+s+s2}{, HP }\PY{l+s+si}{\PYZob{}}\PY{n}{x}\PY{p}{[}\PY{l+m+mi}{1}\PY{p}{]}\PY{l+s+si}{\PYZcb{}}\PY{l+s+s2}{, }\PY{l+s+si}{\PYZob{}}\PY{n}{x}\PY{p}{[}\PY{l+m+mi}{2}\PY{p}{]}\PY{l+s+si}{\PYZcb{}}\PY{l+s+s2}{\PYZdq{}} \PY{k}{for} \PY{n}{x} \PY{o+ow}{in} \PY{n}{data}\PY{p}{]}
    \PY{n}{probs} \PY{o}{=} \PY{p}{[}\PY{n}{x}\PY{p}{[}\PY{l+m+mi}{3}\PY{p}{]} \PY{k}{for} \PY{n}{x} \PY{o+ow}{in} \PY{n}{data}\PY{p}{]}

    \PY{n}{plt}\PY{o}{.}\PY{n}{figure}\PY{p}{(}\PY{n}{figsize}\PY{o}{=}\PY{p}{(}\PY{l+m+mi}{6}\PY{p}{,} \PY{l+m+mi}{5}\PY{p}{)}\PY{p}{)}
    \PY{n}{plt}\PY{o}{.}\PY{n}{barh}\PY{p}{(}\PY{n}{labels}\PY{p}{,} \PY{n}{probs}\PY{p}{,} \PY{n}{edgecolor}\PY{o}{=}\PY{l+s+s2}{\PYZdq{}}\PY{l+s+s2}{black}\PY{l+s+s2}{\PYZdq{}}\PY{p}{,} \PY{n}{color}\PY{o}{=}\PY{l+s+s2}{\PYZdq{}}\PY{l+s+s2}{skyblue}\PY{l+s+s2}{\PYZdq{}}\PY{p}{)}

    \PY{n}{plt}\PY{o}{.}\PY{n}{xlabel}\PY{p}{(}\PY{l+s+s2}{\PYZdq{}}\PY{l+s+s2}{Probabilidad de Captura}\PY{l+s+s2}{\PYZdq{}}\PY{p}{)}
    \PY{n}{plt}\PY{o}{.}\PY{n}{ylabel}\PY{p}{(}\PY{l+s+s2}{\PYZdq{}}\PY{l+s+s2}{Combinaciones}\PY{l+s+s2}{\PYZdq{}}\PY{p}{)}
    \PY{n}{plt}\PY{o}{.}\PY{n}{title}\PY{p}{(}\PY{l+s+sa}{f}\PY{l+s+s2}{\PYZdq{}}\PY{l+s+s2}{Mejores Combinaciones para Captura de }\PY{l+s+si}{\PYZob{}}\PY{n}{pokemon}\PY{o}{.}\PY{n}{capitalize}\PY{p}{(}\PY{p}{)}\PY{l+s+si}{\PYZcb{}}\PY{l+s+s2}{\PYZdq{}}\PY{p}{)}

    \PY{n}{plt}\PY{o}{.}\PY{n}{grid}\PY{p}{(}\PY{n}{axis}\PY{o}{=}\PY{l+s+s2}{\PYZdq{}}\PY{l+s+s2}{x}\PY{l+s+s2}{\PYZdq{}}\PY{p}{,} \PY{n}{linestyle}\PY{o}{=}\PY{l+s+s2}{\PYZdq{}}\PY{l+s+s2}{\PYZhy{}\PYZhy{}}\PY{l+s+s2}{\PYZdq{}}\PY{p}{,} \PY{n}{alpha}\PY{o}{=}\PY{l+m+mf}{0.7}\PY{p}{)}
    \PY{n}{plt}\PY{o}{.}\PY{n}{xlim}\PY{p}{(}\PY{l+m+mi}{0}\PY{p}{,} \PY{l+m+mi}{110}\PY{p}{)}

    \PY{k}{for} \PY{n}{i}\PY{p}{,} \PY{n}{rate} \PY{o+ow}{in} \PY{n+nb}{enumerate}\PY{p}{(}\PY{n}{data}\PY{p}{)}\PY{p}{:}
        \PY{n}{plt}\PY{o}{.}\PY{n}{text}\PY{p}{(}\PY{n}{rate}\PY{p}{[}\PY{l+m+mi}{3}\PY{p}{]}\PY{p}{,} \PY{n}{i}\PY{p}{,} \PY{l+s+sa}{f}\PY{l+s+s2}{\PYZdq{}}\PY{l+s+s2}{ }\PY{l+s+si}{\PYZob{}}\PY{n}{rate}\PY{p}{[}\PY{l+m+mi}{3}\PY{p}{]}\PY{l+s+si}{:}\PY{l+s+s2}{.0f}\PY{l+s+si}{\PYZcb{}}\PY{l+s+s2}{\PYZpc{}}\PY{l+s+s2}{\PYZdq{}}\PY{p}{,} \PY{n}{ha}\PY{o}{=}\PY{l+s+s2}{\PYZdq{}}\PY{l+s+s2}{left}\PY{l+s+s2}{\PYZdq{}}\PY{p}{,} \PY{n}{va}\PY{o}{=}\PY{l+s+s2}{\PYZdq{}}\PY{l+s+s2}{center}\PY{l+s+s2}{\PYZdq{}}\PY{p}{)}

    \PY{n}{plt}\PY{o}{.}\PY{n}{tight\PYZus{}layout}\PY{p}{(}\PY{p}{)}
    \PY{n}{plt}\PY{o}{.}\PY{n}{show}\PY{p}{(}\PY{p}{)}
\end{Verbatim}
\end{tcolorbox}

    \begin{center}
    \adjustimage{max size={0.9\linewidth}{0.9\paperheight}}{output_38_0.png}
    \end{center}
    { \hspace*{\fill} \\}

    \begin{center}
    \adjustimage{max size={0.9\linewidth}{0.9\paperheight}}{output_38_1.png}
    \end{center}
    { \hspace*{\fill} \\}

    Como podemos ver para el caso del \textbf{Onix}, la mejor combinación es
un bajo nivel de vida (1\%) y utilizar la \textbf{Ultraball}. Luego,
como se puede ver, que el \textbf{Pokemon} se encuentre en estado de
\emph{sueño} o \emph{congelamiento} afectan positivamente a la
efectividad de la captura siendo las combinaciones más efectivas las
siguientes:

\begin{itemize}
\tightlist
\item
  Sueño - 1\% HP - Ultraball
\item
  Congelamiento - 1\% HP - Ultraball
\end{itemize}

Por otro lado, para el caso de \textbf{Mewtwo}, el condicionante más
relevante es la utilización de la \textbf{Fastball}. Luego, lógicamente
además de la \textbf{Pokeball} utilizada, el hecho de que tenga poca
vida, aumenta la efectividad de la captura, sin embargo, los efectos de
los puntos de vida restantes no son tan relevantes. El efecto de
\emph{parálisis} es el que más aumenta la efectividad, luego viene
\emph{congelamiento}. Por lo tanto las combinaciones más efectivas son:

\begin{itemize}
\tightlist
\item
  Parálisis - 1\% HP - Fastball
\item
  Congelamiento - 25\% HP - Fastball
\item
  Congelamiento - 1\% HP - Fastball
\end{itemize}

    \subsection{Punto 2e}\label{punto-2e}

Ya comparamos la combinación de efectos de salud, niveles de vida y
tipos de \textbf{Pokeballs}. Veamos ahora que pasa si agregamos a la
combinación, el nivel del \textbf{Pokemon}.

    \begin{tcolorbox}[breakable, size=fbox, boxrule=1pt, pad at break*=1mm,colback=cellbackground, colframe=cellborder]
\prompt{In}{incolor}{20}{\boxspacing}
\begin{Verbatim}[commandchars=\\\{\}]
\PY{n}{results} \PY{o}{=} \PY{p}{\PYZob{}}\PY{p}{\PYZcb{}}

\PY{k}{for} \PY{n}{pokemon\PYZus{}name} \PY{o+ow}{in} \PY{p}{[}\PY{l+s+s2}{\PYZdq{}}\PY{l+s+s2}{onix}\PY{l+s+s2}{\PYZdq{}}\PY{p}{,} \PY{l+s+s2}{\PYZdq{}}\PY{l+s+s2}{mewtwo}\PY{l+s+s2}{\PYZdq{}}\PY{p}{]}\PY{p}{:}
    \PY{n}{results}\PY{p}{[}\PY{n}{pokemon\PYZus{}name}\PY{p}{]} \PY{o}{=} \PY{p}{[}\PY{p}{]}

    \PY{k}{for} \PY{n}{effect} \PY{o+ow}{in} \PY{n}{effects}\PY{p}{:}
        \PY{k}{for} \PY{n}{health} \PY{o+ow}{in} \PY{n}{healths}\PY{p}{:}
            \PY{k}{for} \PY{n}{level} \PY{o+ow}{in} \PY{n}{levels}\PY{p}{:}
                \PY{k}{for} \PY{n}{ball} \PY{o+ow}{in} \PY{n}{pokeballs}\PY{p}{:}
                    \PY{n}{pokemon} \PY{o}{=} \PY{n}{factory}\PY{o}{.}\PY{n}{create}\PY{p}{(}\PY{n}{pokemon\PYZus{}name}\PY{p}{,} \PY{n+nb}{int}\PY{p}{(}\PY{n}{level}\PY{p}{)}\PY{p}{,} \PY{n}{effect}\PY{p}{,} \PY{n+nb}{float}\PY{p}{(}\PY{n}{health}\PY{p}{)}\PY{p}{)}
                    \PY{n}{probs} \PY{o}{=} \PY{p}{[}\PY{p}{]}

                    \PY{k}{for} \PY{n}{\PYZus{}} \PY{o+ow}{in} \PY{n+nb}{range}\PY{p}{(}\PY{l+m+mi}{100}\PY{p}{)}\PY{p}{:}
                        \PY{n}{success}\PY{p}{,} \PY{n}{\PYZus{}} \PY{o}{=} \PY{n}{attempt\PYZus{}catch}\PY{p}{(}\PY{n}{pokemon}\PY{p}{,} \PY{n}{ball}\PY{p}{)}
                        \PY{n}{probs}\PY{o}{.}\PY{n}{append}\PY{p}{(}\PY{n}{success}\PY{p}{)}

                    \PY{n}{results}\PY{p}{[}\PY{n}{pokemon\PYZus{}name}\PY{p}{]}\PY{o}{.}\PY{n}{append}\PY{p}{(}\PY{p}{(}\PY{n}{effect}\PY{o}{.}\PY{n}{name}\PY{p}{,} \PY{n}{health}\PY{p}{,} \PY{n}{level}\PY{p}{,} \PY{n}{ball}\PY{p}{,} \PY{n}{np}\PY{o}{.}\PY{n}{mean}\PY{p}{(}\PY{n}{probs}\PY{p}{)} \PY{o}{*} \PY{l+m+mi}{100}\PY{p}{)}\PY{p}{)}
\end{Verbatim}
\end{tcolorbox}

    \begin{tcolorbox}[breakable, size=fbox, boxrule=1pt, pad at break*=1mm,colback=cellbackground, colframe=cellborder]
\prompt{In}{incolor}{21}{\boxspacing}
\begin{Verbatim}[commandchars=\\\{\}]
\PY{k}{for} \PY{n}{pokemon}\PY{p}{,} \PY{n}{result\PYZus{}data} \PY{o+ow}{in} \PY{n}{results}\PY{o}{.}\PY{n}{items}\PY{p}{(}\PY{p}{)}\PY{p}{:}
    \PY{n}{data} \PY{o}{=} \PY{n+nb}{sorted}\PY{p}{(}\PY{n}{result\PYZus{}data}\PY{p}{,} \PY{n}{key}\PY{o}{=}\PY{k}{lambda} \PY{n}{x}\PY{p}{:} \PY{n}{x}\PY{p}{[}\PY{l+m+mi}{4}\PY{p}{]}\PY{p}{)}\PY{p}{[}\PY{o}{\PYZhy{}}\PY{l+m+mi}{10}\PY{p}{:}\PY{p}{]}
    \PY{n}{labels} \PY{o}{=} \PY{p}{[}\PY{l+s+sa}{f}\PY{l+s+s2}{\PYZdq{}}\PY{l+s+si}{\PYZob{}}\PY{n}{x}\PY{p}{[}\PY{l+m+mi}{0}\PY{p}{]}\PY{l+s+si}{\PYZcb{}}\PY{l+s+s2}{, HP }\PY{l+s+si}{\PYZob{}}\PY{n}{x}\PY{p}{[}\PY{l+m+mi}{1}\PY{p}{]}\PY{l+s+si}{\PYZcb{}}\PY{l+s+s2}{, LVL }\PY{l+s+si}{\PYZob{}}\PY{n}{x}\PY{p}{[}\PY{l+m+mi}{2}\PY{p}{]}\PY{l+s+si}{\PYZcb{}}\PY{l+s+s2}{, }\PY{l+s+si}{\PYZob{}}\PY{n}{x}\PY{p}{[}\PY{l+m+mi}{3}\PY{p}{]}\PY{l+s+si}{\PYZcb{}}\PY{l+s+s2}{\PYZdq{}} \PY{k}{for} \PY{n}{x} \PY{o+ow}{in} \PY{n}{data}\PY{p}{]}
    \PY{n}{probs} \PY{o}{=} \PY{p}{[}\PY{n}{x}\PY{p}{[}\PY{l+m+mi}{4}\PY{p}{]} \PY{k}{for} \PY{n}{x} \PY{o+ow}{in} \PY{n}{data}\PY{p}{]}

    \PY{n}{plt}\PY{o}{.}\PY{n}{figure}\PY{p}{(}\PY{n}{figsize}\PY{o}{=}\PY{p}{(}\PY{l+m+mi}{6}\PY{p}{,} \PY{l+m+mi}{5}\PY{p}{)}\PY{p}{)}
    \PY{n}{plt}\PY{o}{.}\PY{n}{barh}\PY{p}{(}\PY{n}{labels}\PY{p}{,} \PY{n}{probs}\PY{p}{,} \PY{n}{edgecolor}\PY{o}{=}\PY{l+s+s2}{\PYZdq{}}\PY{l+s+s2}{black}\PY{l+s+s2}{\PYZdq{}}\PY{p}{,} \PY{n}{color}\PY{o}{=}\PY{l+s+s2}{\PYZdq{}}\PY{l+s+s2}{skyblue}\PY{l+s+s2}{\PYZdq{}}\PY{p}{)}

    \PY{n}{plt}\PY{o}{.}\PY{n}{xlabel}\PY{p}{(}\PY{l+s+s2}{\PYZdq{}}\PY{l+s+s2}{Probabilidad de Captura}\PY{l+s+s2}{\PYZdq{}}\PY{p}{)}
    \PY{n}{plt}\PY{o}{.}\PY{n}{ylabel}\PY{p}{(}\PY{l+s+s2}{\PYZdq{}}\PY{l+s+s2}{Combinaciones}\PY{l+s+s2}{\PYZdq{}}\PY{p}{)}
    \PY{n}{plt}\PY{o}{.}\PY{n}{title}\PY{p}{(}\PY{l+s+sa}{f}\PY{l+s+s2}{\PYZdq{}}\PY{l+s+s2}{Mejores Combinaciones para Captura de }\PY{l+s+si}{\PYZob{}}\PY{n}{pokemon}\PY{o}{.}\PY{n}{capitalize}\PY{p}{(}\PY{p}{)}\PY{l+s+si}{\PYZcb{}}\PY{l+s+s2}{\PYZdq{}}\PY{p}{)}

    \PY{n}{plt}\PY{o}{.}\PY{n}{grid}\PY{p}{(}\PY{n}{axis}\PY{o}{=}\PY{l+s+s2}{\PYZdq{}}\PY{l+s+s2}{x}\PY{l+s+s2}{\PYZdq{}}\PY{p}{,} \PY{n}{linestyle}\PY{o}{=}\PY{l+s+s2}{\PYZdq{}}\PY{l+s+s2}{\PYZhy{}\PYZhy{}}\PY{l+s+s2}{\PYZdq{}}\PY{p}{,} \PY{n}{alpha}\PY{o}{=}\PY{l+m+mf}{0.7}\PY{p}{)}
    \PY{n}{plt}\PY{o}{.}\PY{n}{xlim}\PY{p}{(}\PY{l+m+mi}{0}\PY{p}{,} \PY{l+m+mi}{110}\PY{p}{)}

    \PY{k}{for} \PY{n}{i}\PY{p}{,} \PY{n}{rate} \PY{o+ow}{in} \PY{n+nb}{enumerate}\PY{p}{(}\PY{n}{data}\PY{p}{)}\PY{p}{:}
        \PY{n}{plt}\PY{o}{.}\PY{n}{text}\PY{p}{(}\PY{n}{rate}\PY{p}{[}\PY{l+m+mi}{4}\PY{p}{]}\PY{p}{,} \PY{n}{i}\PY{p}{,} \PY{l+s+sa}{f}\PY{l+s+s2}{\PYZdq{}}\PY{l+s+s2}{ }\PY{l+s+si}{\PYZob{}}\PY{n}{rate}\PY{p}{[}\PY{l+m+mi}{4}\PY{p}{]}\PY{l+s+si}{:}\PY{l+s+s2}{.0f}\PY{l+s+si}{\PYZcb{}}\PY{l+s+s2}{\PYZpc{}}\PY{l+s+s2}{\PYZdq{}}\PY{p}{,} \PY{n}{ha}\PY{o}{=}\PY{l+s+s2}{\PYZdq{}}\PY{l+s+s2}{left}\PY{l+s+s2}{\PYZdq{}}\PY{p}{,} \PY{n}{va}\PY{o}{=}\PY{l+s+s2}{\PYZdq{}}\PY{l+s+s2}{center}\PY{l+s+s2}{\PYZdq{}}\PY{p}{)}

    \PY{n}{plt}\PY{o}{.}\PY{n}{tight\PYZus{}layout}\PY{p}{(}\PY{p}{)}
    \PY{n}{plt}\PY{o}{.}\PY{n}{show}\PY{p}{(}\PY{p}{)}
\end{Verbatim}
\end{tcolorbox}

    \begin{center}
    \adjustimage{max size={0.9\linewidth}{0.9\paperheight}}{output_42_0.png}
    \end{center}
    { \hspace*{\fill} \\}

    \begin{center}
    \adjustimage{max size={0.9\linewidth}{0.9\paperheight}}{output_42_1.png}
    \end{center}
    { \hspace*{\fill} \\}

    Como se puede observar, el hecho de variar el nivel del \textbf{Pokemon}
afecta a la efectividad de captura. Evidentemente a niveles más bajos,
la efectividad es más alta, esto se puede ver evidentemente tanto para
\textbf{Onix} como para \textbf{Mewtwo}. Por lo tanto si se puede
concluir que es efectiva la variación del nivel a un valor menor que el
máximo.


    % Add a bibliography block to the postdoc



\end{document}
